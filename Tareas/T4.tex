\documentclass[10pt]{article}

\usepackage[margin=.7in]{geometry}
\usepackage{enumitem}
\usepackage{multicol}
\usepackage{float}
\usepackage{multicol}
\usepackage{tikz,pgfplots}
\usepackage{amsmath,amsthm,amssymb}
\usepackage{mathtools}
\usepackage[utf8]{inputenc}
\usepackage[spanish]{babel}
\usetikzlibrary{babel}

\usepackage{mdframed}

\usepackage{amstext} % for \text macro
\usepackage{array}   % for \newcolumntype macro
\newcolumntype{C}{>{$}c<{$}} % math-mode version of "c" column type
\newcolumntype{L}{>{$}l<{$}} % math-mode version of "l" column type

\newcounter{problemctr}[section]
\newenvironment{problema}{
    \refstepcounter{problemctr}
    \mdfsetup{%
        frametitle={\tikz\node[fill=white,rectangle,inner sep=1pt,outer
        sep=1pt]{Problema \theproblemctr};},
        frametitleaboveskip=-0.5\ht\strutbox,
        frametitlealignment=\center
    }%
    \begin{mdframed}[style=exampledefault]
    }{\end{mdframed}
}


\usepackage[T1]{fontenc}

% Resize abs and norm
\DeclarePairedDelimiter{\abs}{\lvert}{\rvert}
\DeclarePairedDelimiter{\norm}{\|}{\|}
\makeatletter
\let\oldabs\abs
\def\abs{\@ifstar{\oldabs}{\oldabs*}}
\let\oldnorm\norm
\def\norm{\@ifstar{\oldnorm}{\oldnorm*}}
\makeatother

\DeclareMathOperator\dom{dom}

\begin{document}
{\sc Tarea 4 - EDO\hfill Sebastián Sánchez}

\begin{center}
\textbf{Agradecimientos:} Camila Guajardo Vásquez, Pablo Navarro, Felipe
Gonzáles, Claudio Melendez.
\end{center}

% P1
\begin{problema}
{\hfill G. Teschl - Problem 6.8}\\
	Sea \(\phi(t)\)	la solución a un sistema autónomo de primer orden. Suponga
	que \(\lim_{t\to \infty} \phi(t) = x\in M\). Muestre que \(x\) es un punto
	fijo y que \(\lim_{t\to \infty} \dot{\phi}(t) = 0\).
\end{problema}
Dado que \(\phi(t)\) es solución, sabemos que \(\dot{\phi}(t) = f(\phi(t))\).
Tomando límite y usando que \(f\) es continua, obtenemos que
\[
	\lim_{t\to \infty} \dot{\phi}(t)
	=
	\lim_{t\to \infty} f(\phi(t))
	=
	f(x)
.\]
Así que si \(x\) es un punto fijo se concluye que \(\lim_{t\to \infty}
\dot{\phi}(t) = 0\). Supongamos que \(x\) no es un punto fijo, entonces,
\(\lim_{t\to \infty} \dot{\phi}(t) \ne 0\). En particular, existe un \(\epsilon > 0\)
y \(T>0\) tal que para todo \(t\ge T\) se tiene que \(\abs{\dot{\phi(t)}} >
\epsilon\). Por otro lado, \(\phi(t) \xrightarrow{t\to \infty} x\), así que
existe \(S > 0\) tal que para todo \(t, s > S\) se cumple que \(\abs{\phi(t) -
\phi(s)} < \epsilon\). Luego, por TVM y para \(T, S\) suficientemente grandes tenemos que
\[
	\epsilon > \abs{\phi(t) - \phi(s)} = \abs{\dot\phi(c)} \abs{t-s} \ge \epsilon \abs{t-s}
.\]
El lado derecho diverge y está acotado. Contradicción.
Concluimos entonces que \(x\) debe ser un punto fijo.

% P2
\begin{problema}
{\hfill G. Teschl - Problem 6.10}\\
	Un punto \(x\in M\) se dice \textbf{no errante} si en toda vecindad \(U\) de
	\(x\) existe una sucesión \(t_n\to \infty\) de tiempos positivos tal que
	\(\Phi_{t_n}(U)\cap U \ne \varnothing\) para todo \(t_n\). El conjunto de
	los puntos no errantes se denota por \(\Omega(f)\).
	\begin{enumerate}[label=(\roman*)]
		\item \(\Omega(f)\) es un conjunto cerrado e invariante.
		(\textit{hint:} Muestre que es el complemento de un abierto)

		\item \(\Omega(f)\) contiene todas las órbitas periódicas (incluyendo
		los puntos fijos).

		\item \(\omega_{+}(x) \subseteq \Omega(f)\) para todo \(x\in M\).
	\end{enumerate}
	Encuentre el conjunto de los puntos no errantes \(\Omega(f)\) para el
	sistema \(f(x,y) = (y,-x)\).
\end{problema}
\begin{enumerate}[label=(\roman*)]
	\item
	Supongamos que \(x\not\in \Omega(f)\). Probaremos que existe una vecindad
	\(V\) de \(x\) tal que \(V\cap \Omega(f) = \varnothing\).

	Como \(x\not\in\Omega(f)\), existe una vecindad \(V\) de \(x\) tal que para toda secuencia
	de tiempos \(t_n\to \infty\) se cumple que \(\Phi_{t_n} (V) \cap V =
	\varnothing\). Supongamos que \(y \in V\cap\Omega(f)\), entonces existe una
	vecindad \(U \subset V\cap\Omega(f)\) y una secuencia de tiempos \(s_n\to
	\infty\) tal que \(\Phi_{s_n}(U)\cap U \ne \varnothing\). En particular,
	\(\varnothing \ne \Phi_{s_n}(V\cap U) \cap U \subset
	\Phi_{s_n}(V)\cap V = \varnothing\). Contradicción. Luego, \(V\cap\Omega(f)
	= \varnothing\) y se concluye que \(\Omega(f)\) es cerrado.

	Para ver que es invariante, consideremos \(y = \Phi_{s}(x) \in \gamma(x)\) y \(V\) una
	vecindad de \(y\). Luego, \(U = \Phi_{s-s=0}(V)\) es una vecindad de \(x\). Dado
	que \(x\in \Omega(f)\) entonces existe una sucesión de tiempos \(t_n\to \infty\)
	tal que \(\Phi_{t_n}(U) \cap U \ne \varnothing\). Aplicando \(\Phi_{s}\)
	tenemos que \(\Phi_s(\Phi_{t_n}(U) \cap U) \ne \varnothing\) y en particular
	\(\varnothing \ne \Phi_{s+t_n}(U) \cap \Phi_s(U) = \Phi_{t_n}(V) \cap V \).
	Se concluye entonces que \(y\in \Omega(f)\).


	\item Supongamos que \(x\) es un punto con órbita periódica. Luego, como
	\(\Phi\colon I_x\times \left\{ x \right\}\to M\) es homeomorfismo, las
	imágenes de abiertos son abiertos. En particular, para toda vecindad \(U\)
	de \(x\) se tiene que \(\Phi_{nT(x)}(U) \cap U \ne \varnothing\) para todo
	\(n\ge 0\). Tomando \(t_n = nT\to \infty\) se concluye que \(\gamma(x)
	\subset \Omega(f)\).

	Para los puntos fijos, cualquier sucesión de tiempos \(t_n \to \infty\)
	sirve.

	\item Sea \(y\in \omega_{+}(x)\). Sea \(y\in V\) una vecindad de \(y\).
	Luego, para toda sucesión \(t_n \to \infty\) existe un \(N\in \mathbb{N}\) tal
	que \(\Phi_{t_n}(x)\in V\) para todo \(n\ge N\). En particular,
	\(\Phi_{t_n}(V) \cap V \ne \varnothing\) para todo \(n\ge N\), así que
	\(y\in \Omega(f)\).
\end{enumerate}

Consideremos ahora el sistema dado por \((\dot{x},\dot{y}) = (y,-x)\). El único
punto fijo es el \((0,0)\). Además, si \(\dot{x} = 0\) entonces \(y = x = 0\) y
si \(\dot{y} = 0\) entonces \(y = x = 0\). Por lo tanto, no hay curvas que sean
completamente verticales ni horizontales. Consideremos el sistema matricial
asociado
\[
	\begin{pmatrix} \dot{x} \\ \dot{y} \end{pmatrix}
	=
	\begin{pmatrix}
	0 & 1\\
	-1 & 0
	\end{pmatrix}
	\begin{pmatrix} x \\ y \end{pmatrix}
.\]
Los valores propios son \(\pm i\), así que las soluciones son
periódicas. Se concluye entonces que \(\Omega(f) = \mathbb{R}^2\).

% P3
\begin{problema}
{\hfill G. Teschl - Problem 6.16}\\
	Considere el sistema
	\begin{align*}
		\dot{x} &= x - y - x(x^2+y^2) + \frac{xy}{\sqrt{x^2+y^2}}\\
		\dot{y} &= x + y - y(x^2+y^2) - \frac{x^2}{\sqrt{x^2+y^2}}
	\end{align*}
	Muestre que \((1,0)\) no es estable aunque satisface
	\[
		\lim_{t\to \infty} \abs{\phi(t,x) - (1,0)} = 0
		\quad\forall x\in U((1,0))
	.\]

	\textit{Hint}: Muestre que el sistema en coordenadas polares está dado por
	\(\dot{r} = r(1-r^2)\), \(\dot{\theta} = 2\sin^2(\theta/2)\).
\end{problema}
Sea \(x=r\cos(\theta)\) y \(y = r\sin(\theta)\). Luego,
\[
	r^2 = x^2 + y^2
	\Rightarrow
	\dot{r} r = x\dot{x} + y\dot{y}
	\implies
	\dot{r} r = r^3 - r^4
.\]
Por otro lado,
\[
	\tan(\theta) = \frac{y}{x}
	\Rightarrow
	\dot{\theta} \sec^2(\theta) = \frac{\dot{y}x - \dot{x}y}{x^2}
	\Rightarrow
	\dot{\theta} = 1 - \cos(\theta) = 2\sin^2(\theta/2)
.\]
En resumen, el sistema equivalente es
\[
	\dot r = r(1-r^2) \hspace{.5cm} \dot \theta = 2\sin^2(\theta/2)
.\]

Evaluando en \(1,0)\), vemos que es un punto fijo. Analizando las nullclinas
(en las zonas de interés) tenemos que
\[
	\dot r = 0 \iff r = 0 \lor r = 1
	\hspace{2cm}
	\dot \theta = 0 \iff \theta = 4\pi \lor \theta = 0
.\]

Miremos la vecindad del punto \((1,0)\). Si \(r < 1\), \(\dot r > 0\) así que las
soluciones se alejan del origen. Lo contrario pasa para \(r> 1\), aquí \(\dot r
< 0\) y las soluciones se acercan al origen. En \(r=1\) la solución se mantiene
a una distancia constante. Por otro lado, \(\dot \theta > 0\) así
que las soluciones giran de manera antihoraria. La intuición es que las
soluciones no son estables, pues se escapan de cualquier vecindad de \((1,0)\)
pero órbitan asintóticamente a la órbita circular.

Para probar la intuición, usaremos el teorema de Poincaré-Bendixson.
Consideremos la región \(1 \le r \le 1+\epsilon\) para \(\epsilon > 0\)
suficientemente pequeño, dado que \(\dot r < 0\) las soluciones se mantienen
encerradas en el anillo (no pueden pasar por \(r=1\) pues las trayectorias
serían perpendiculares con el campo vectorial) y no hay puntos fijos en el
interior de la región, se concluye por el teorema que las soluciones tienden a una órbita
cerrada. De esta forma, las soluciones dan vuelta en anillo y vuelven al \((1,0)\).
El análisis para \(1-\epsilon \le r \le 1\) es análogo.

% P4
\begin{problema}
	Considere el sistema
	\[
		\dot{x} = x^2 - x - y, \quad \dot{y} = x
	.\]
	\begin{enumerate}[label=(\alph*)]
		\item Demuestre que el origen es \textit{asintóticamente} estable
		construyendo una función de Liapunov estricta.

		\item Demuestre que el origen es, de hecho, \textit{exponencialmente}
		estable a través del criterio de linealización.
	\end{enumerate}
\end{problema}
Consideremos una función de la forma \(L(x,y) = x^2 + axy + by^2\), queremos que
la función \(L\) tenga un mínimo local en el origen, por lo que
\[
	DL(x,y) = (2x+ay, ax + 2by)
	\hspace{1cm}
	D^2 L(x,y) =
	\begin{pmatrix}
	2 & a\\ a & 2b
	\end{pmatrix}
\]
Deben cumplir con \(DL(0,0) = 0\) y \(D^2L(x,y)\) debe ser definida positiva. Por el criterio
de Sylvester, \(4b-a^2 > 0\) garantiza lo segundo. Eligiendo \(a=1\) podemos
tomar \(b = 3/4\) y la desigualdad sigue valiendo. Luego, una función de Liapunov es
\(L(x,y) = x^2 + xy + (3/4)y^2\) que es estricta por la elección de \(a\) y \(b\).


Viendo el sistema linealizado, tenemos que
\[
	Df(x,y) =
	\begin{pmatrix}
	2x-1 & -1\\ 1 & 0
	\end{pmatrix}
	\implies
	Df(0,0)
	=
	\begin{pmatrix}
	-1 & -1\\ 1 & 0
	\end{pmatrix}
.\]
Los valores propios son \(\frac{-1\pm \sqrt{-3}}{2}\), que tienen parte real
negativa, así que el punto fijo es exponencialmente estable.

% P5
\begin{problema}
	El péndulo con fricción está descrito por
	\[
		\ddot x = - \eta \dot x - \sin(x), \quad \eta > 0
	.\]
	¿Está la energía conservada en este caso? Muestre que la energía puede ser
	usada como una función de Liapunov y pruebe que el punto fijo \((\dot x,
	x) = (0,0)\) es (\textit{asintóticamente}) estable. ¿Cómo cambia el retrato de fase?
\end{problema}
La ecuación de la energía es
\[
	E(x,\dot x) = \frac{\dot{x}^2}{2} + 1 - \cos(x)
.\]
Derivando con respecto a \(t\) vemos que
\[
	\dot{E} = \dot{x}\ddot{x} + \dot{x} \sin(x) = -\eta \dot{x}^2
.\]
Como \(E(0,0) = 0\) y \(E(x,\dot{x}) > 0\) para \((x,\dot x) \ne 0\). Concluimos
que \(E\) es una función de Liapunov para el sistema. Por otro lado, el computo
anterior muestra que la derivada es estrictamente decreciente, así que no pueden
haber órbitas periódicas regulares. Se sigue que el sistema es asintóticamente
estable.

Para ver el retrato de fase, primero escribamos el sistema como uno de primer
orden de la siguiente forma:
\[
	\dot x = y \hspace{1cm} \dot y = -\eta y - \sin(x)
.\]
El Jacobiano del sistema es
\[
	Df(x,y)
	=
	\begin{pmatrix}
	0 & 1\\
	-\cos(x) & -\eta
	\end{pmatrix}
	\Rightarrow
	Df(0,0)
	=
	\begin{pmatrix}
	0 & 1\\
	-1 & -\eta
	\end{pmatrix}
.\]
Viendo el polinomio característico, los valores propios están dados por
\[
	\lambda = \frac{-\eta \pm \sqrt{\eta^2 - 4}}{2}
.\]
Analicemos los distintos casos.
\begin{itemize}
	\item Si \(\eta \in (0,2)\), entonces los valores propios son complejos con
	parte real negativa, por lo tanto el origen es un sumidero espiral.

	\item Si \(\eta = 2\), entonces \(\lambda = \pm 1\) y el origen es una
	silla.

	\item Si \(\eta > 2\), entonces los valores propios son reales negativos,
	así que el origen es un sumidero.
\end{itemize}

% P6
\begin{problema}
	Analice el sistema competitivo dado por
	\[
	\begin{cases}
		\dot x &= (1-y-\lambda x) x,\\
		\dot y &= \alpha (1-x-\mu y)y
	\end{cases}
	\hspace{1cm}
	\alpha, \lambda, \mu > 0
	.\]
	En el primer cuadrante (\(x,y\ge 0\)) siguiendo la exposición dada en el
	capítulo 7.1 de Teschl.
	\begin{enumerate}
		\item Encuentre conjuntos invariantes.

		\item Determine las dos \textit{nullclinas} y úselas para determinar el
		sentido del campo vectorial en el primer cuadrante \(Q\) (note que las
		nullclinas dividen \(Q\) en regiones donde el signo de cada componente
		del campo vectorial no cambia).

		\item Encuentre los puntos fijos del sistema y estudie su estabilidad
		(en términos de los parámetros del problema).

		\item Dibuje el retrato de fase (\textit{sugerencia:} Lema 7.2 será
		útil).
	\end{enumerate}
\end{problema}
Comenzaremos mirando las nullclinas.
\[
	\begin{cases}
		\dot x = 0 &\iff x = 0 \lor y = 1-\lambda x\\
		\dot y = 0 &\iff y = 0 \lor y = \frac{1-x}{\mu}
	\end{cases}
.\]
Por otro lado, los puntos fijos son \((0,0), (0,1/\mu), (1/\lambda, 0)\) y
\((\frac{\mu-1}{\lambda\mu - 1}, \frac{\lambda - 1}{\lambda\mu -1})\). Notemos que
si \(\lambda \mu > 1\), entonces \(\mu > 1\) y \(\lambda > 1\). Mientras que si
\(\lambda\mu < 1\), entonces \(\mu < 1\) y \(\lambda < 1\).

La derivada de \(f\) está dada por
\[
	Df(x,y) =
	\begin{pmatrix}
	1-y-2\lambda x & -x\\
	-\alpha x & \alpha (1-x-2\mu y)
	\end{pmatrix}
.\]

Evaluando en los puntos fijos
\begin{itemize}
	\item En \(x*=(0,0)\) se tiene que
	\[
	Df(x*) =
	\begin{pmatrix}
	1 & 0\\
	0 & \alpha
	\end{pmatrix}
	.\]
	Como los valores propios son positivos, el punto es una fuente.

	\item En \(x*=(0,1/\mu)\) se tiene que
	\[
	Df(x*) =
	\begin{pmatrix}
	1-\frac{1}{\mu} & 0\\
	0 & -\alpha
	\end{pmatrix}
	.\]
	Aquí, si \(\mu > 1\) los valores propios tienen signo opuesto y tenemos una
	silla. Si \(\mu < 1\) los valores propios negativos y el punto es un
	sumidero.

	\item En \(x*=(1/\lambda, 0)\) se tiene que
	\[
		Df(x*) =
		\begin{pmatrix}
		-1 & -1/\lambda\\
		-\alpha/\lambda & \alpha(1-1/\lambda)
		\end{pmatrix}
	.\]
	Aquí la traza \(T = \alpha - \alpha/\lambda - 1\) y el determinante \(D =
	-\alpha(1 - \frac{1}{\lambda} + \frac{1}{\lambda^2})\). Dado que
	\[
		D = 0 \implies \lambda = \frac{1\pm \sqrt{3}i}{2}
	.\]
	Se deduce que el determinante es siempre negativo. Así que tenemos una
	silla.

	\item En \(x*=(\frac{\mu-1}{\lambda\mu-1},
	\frac{\lambda-1}{\lambda\mu-1})\) se tiene que
	\begin{align*}
		Df(x*)
		&=
		\frac{1}{\lambda\mu-1}
		\begin{pmatrix}
		\lambda\mu-1 - (\lambda-1) - 2\lambda(\mu-1) & -(\mu-1)\\
		-\alpha(\mu-1) & \alpha((\lambda\mu-1) - (\mu-1) - 2\mu(\lambda-1))
		\end{pmatrix}
		\\&=
		\frac{1}{\lambda\mu-1}
		\begin{pmatrix}
		\lambda(1-\mu) & -(\mu-1)\\
		-\alpha(\mu-1) & \alpha\mu(1-\lambda)
		\end{pmatrix}
	\end{align*}
	Después de las manipulaciones, la traza (\(T\)) y el determinante (\(D\))
	dan
	\begin{align*}
	T &= -\alpha \frac{\lambda\mu+1}{\lambda\mu-1} +
	\frac{\alpha\mu+\lambda}{\lambda\mu-1}\\
	D &=
	\left(\frac{1}{\lambda\mu-1}\right)^2 \alpha
	\left[\lambda\mu(1-\mu)(1-\lambda) - (1-\mu)^2\right]
	\\&=
	\left(\frac{1}{\lambda\mu-1}\right)^2 \alpha(1-\mu)
	\left[\lambda\mu(1-\lambda) - (1-\mu)\right]
	\end{align*}
\end{itemize}

% P7
\begin{problema} (\texttt{Oscilador glucolítico})~
	Las células vivas obtiene energía al descomponer el azúcar en un proceso
	llamado \textit{glucólisis}. La glucólisis puede proceder de manera
	oscilatoria, con la concentración de varios compuestos químicos intermedios
	aumentando y disminuyendo en un período de varios minutos. El siguiente
	sistema es un modelo idealizado de la evolución de la concentración \(x_1\)
	de ADP (adenosina difosfato) y la concentración \(x_2\) de F6P
	(fructosa-6-fosfato).
	\[
		\begin{cases}
			\dot x_1 &= -x_1 + ax_2 + x_1^2 x_2\\
			\dot x_2 &= b - ax_2 - x_1^2 x_2
		\end{cases}
		\hspace{1cm}
		a,b>0
	.\]
	En este problema demostrará que para \(a=0.1\) y \(b=0.5\) el sistema admite
	una órbita periódica en el primer cuadrante \(Q\coloneqq \left\{ x_1, x_2 >
	0 \right\}\).

	\begin{enumerate}[label=(\alph*)]
		\item Encuentre el único punto fijo del sistema y demuestre que es una
		fuente.

		\item Encuentre las nullclinas y úselas para dividir el primer cuadrante
		en regiones donde el campo vectorial apunta NE, SE, SO, NO.

		\item Identifique un subconjunto compacto \(\mathcal{R}\) de
		\(\overline{Q}\) que sea \((+)\)-invariante y no tenga puntos fijos.
		Concluya que \(\mathcal{R}\) contiene una órbita periódica.
		(\textit{Sugerencia:} Note que \(f(x)\cdot (1,1) = b - x_1\))
	\end{enumerate}
\end{problema}

Veamos el punto fijo
\[
	f(x) = 0 \iff
	\begin{cases}
	-x_1 + x_2/10 + x_1^2 x_2 &= 0\\
	1/2 - x_2/10 - x_1^2 x_2 &= 0
	\end{cases}
	\iff
	\begin{pmatrix} x_1 \\ x_2 \end{pmatrix}
	=
	\begin{pmatrix} 1/2 \\ 10/7 \end{pmatrix}
.\]
Viendo la derivada de \(f\) en el punto fijo \(x*\) tenemos que
\[
	Df(x*)
	=
	\begin{pmatrix}
	3/7 & 7/20\\
	-10/7 & -7/20
	\end{pmatrix}
.\]
La traza \(T = 11/140\) y el determinante \(D = 7/20\) son positivos y se cumple
que \(T^2 < 4D\), así que el punto es una fuente.

Ahora veamos las nulclinas,
\[
	\begin{cases}
		\dot x_1 = 0 & \iff x_2 = \frac{x_1}{x_1^2 + 1/10} \eqqcolon L_1\\
		\dot x_2 = 0 & \iff x_2 = \frac{1}{2(x_1^2 + 1/10)} \eqqcolon L_2
	\end{cases}
.\]
También miremos los ejes para tener una idea del campo vectorial.
\[
	\begin{cases}
		x_1 = 0&\implies \dot x_1 = x_2/10 \land \dot x_2 = \frac{1}{2} -
		\frac{x_2}{10}\\
		x_2 = 0&\implies \dot x_1 = -x_1 \land \dot x_2 = b
	\end{cases}
.\]
Se sigue que en el eje vertical, el punto \(5\) es un punto estable (debajo de
\(5\) el campo apunta hacia arriba y por sobre \(5\) apunta hacia abajo) y el campo
vectorial apunta hacia la derecha. Mientras tanto, en el eje horizontal, el
campo apunta a la izquierda y arriba.

\begin{figure}[H]
\centering
\begin{tikzpicture}[scale=1.5,domain=0:4, smooth]
	\draw[->] (0,0) -- (4,0) node[below] {\(x_1\)};
	\draw[->] (0,0) -- (0,6) node[left] {\(x_2\)};

	\draw[dashed] (.5, 6) -- (.5,-.1) node[below] {\small \(0.5\)};

	\draw[<-,very thick,purple] (0,5.2) -- (0,6);
	\draw[->,very thick,purple] (0,5.6) -- (.2,5.6);

	\draw[->,very thick,purple] (0,2.5) -- (0,3.5);
	\draw[->,very thick,purple] (0,3) -- (.2,3);

	\draw[<-<,very thick,purple] (.5,0) -- (2.5,0);
	\draw[->,very thick,purple] (1.2,0) -- (1.2,.2);

	\draw[red,thick] plot (\x, {\x/(\x^2+0.1)}) node[right] {\(L_1\)};
	\draw[blue, thick] plot (\x, {1/(2*(\x^2+0.1))}) node[right] {\(L_2\)};

	\node at (1.2,1.5) {\(Q_1\)};
	\node at (.2,1.8) {\small\(Q_2\)};
	\node at (.3,.5) {\(Q_3\)};
	\node at (1.5,.5) {\(Q_4\)};
\end{tikzpicture}
\caption{En la curva roja los vectores son verticales y en la curva azul son
horizontales.}
\end{figure}

Veamos la dirección del campo vectorial en las distintas regiones.
\begin{itemize}
	\item \(Q_1\): \(\dot y < 0\) y \(\dot x > 0\), así que apunta \(SE\).
	\item \(Q_2\): \(\dot y > 0\) y \(\dot x > 0\), así que apunta \(NE\).
	\item \(Q_3\): \(\dot y > 0\) y \(\dot x < 0\), así que apunta \(NO\).
	\item \(Q_4\): \(\dot y < 0\) y \(\dot x < 0\), así que apunta \(SO\).
\end{itemize}

Para \(x_1 \to \infty\), vemos que \(\dot x_1 \approx x_1^2\) y \(\dot x_2
\approx -x_1^2\), así que \(dx_1/dx_2 \approx -1\). Esto sugiere que el campo
vectorial eventualmente apunta SE. Comprobando la intuición vemos que
\[
	f(x)\cdot (1,1)
	=
	(-x_1 + x_2/10 + x_1^2 x_2, 1/2 - x_2/10 - x_1^2 x_2)\cdot(1,1)
	=
	1/2 - x_1
,\]
y por lo tanto para \(x_1 = 1/2\) el campo vectorial tiene pendiente SE, más
aún, para \(x_1 > 1/2\) las trayectorias tienen pendiente menor a \(-1\), así
que cualquier recta \(L\) de pendiente \(-1\) que esté por sobre el punto fijo tenemos
una región \(R\) atrapadora delimitada por la recta \(L\), la vertical \(x_1 =
1/2\) y la curva \(L_2\) (en azul); La cual es compacta (cerrada y acotada),
(\(+\))-invariante, y no tiene puntos fijos. Por Poincaré-Bendixson se concluye que
\(R\) contiene órbitas periodicas.
\end{document}
