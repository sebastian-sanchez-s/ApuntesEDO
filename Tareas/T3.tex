\documentclass[10pt]{article}

\usepackage[margin=.7in]{geometry}
\usepackage{enumitem}
\usepackage{multicol}
\usepackage{float}
\usepackage{multicol}
\usepackage{tikz,pgfplots}
\usepackage{amsmath,amsthm,amssymb}
\usepackage{mathtools}
\usepackage[utf8]{inputenc}
\usepackage[spanish]{babel}
\usetikzlibrary{babel}

\usepackage{mdframed}

\usepackage{amstext} % for \text macro
\usepackage{array}   % for \newcolumntype macro
\newcolumntype{C}{>{$}c<{$}} % math-mode version of "c" column type
\newcolumntype{L}{>{$}l<{$}} % math-mode version of "l" column type

\newcounter{problemctr}[section]
\newenvironment{problema}{
    \refstepcounter{problemctr}
    \mdfsetup{%
        frametitle={\tikz\node[fill=white,rectangle,inner sep=1pt,outer
        sep=1pt]{Problema \theproblemctr};},
        frametitleaboveskip=-0.5\ht\strutbox,
        frametitlealignment=\center
    }%
    \begin{mdframed}[style=exampledefault]
    }{\end{mdframed}
}


\usepackage[T1]{fontenc}

% Resize abs and norm
\DeclarePairedDelimiter{\abs}{\lvert}{\rvert}
\DeclarePairedDelimiter{\norm}{\|}{\|}
\makeatletter
\let\oldabs\abs
\def\abs{\@ifstar{\oldabs}{\oldabs*}}
\let\oldnorm\norm
\def\norm{\@ifstar{\oldnorm}{\oldnorm*}}
\makeatother

\DeclareMathOperator\dom{dom}

\begin{document}
{\sc Tarea 3 - EDO\hfill Sebastián Sánchez}

\begin{center}
\textbf{Agradecimientos:} Pablo Navarro.
\end{center}

%P1
\begin{problema}
    Sean \(A,B \in \mathbb{R}^{n\times n}\). Demuestre que el sistema lineal
    \[
        \dot{x} = (\sin(t) A + e^{-t} B)x
    \]
    es estable, es decir, que todas sus soluciones permanecen acotadas para
    \(t\ge 0\).
\end{problema}
% R1
Consideremos el sistema no perturbado \(\dot{x} = \sin(t) A x\). Notemos que es
periódico con \(T=2\pi\) y tiene coeficientes continuos.
Sabemos que las soluciones a este sistema son de la forma:
\[
    x(t) = \Pi(t,t_0) K,\quad K\in \mathbb{R}
.\]
Donde \(\Pi(t,t_0)\) denota al propagador. Por el teorema de Floquet se tiene:
\[
    \Pi(t,t_0) = P(t,t_0) \exp\left((t-t_0)Q(t_0)\right)
\]
y sabemos que \(\norm{P(t,t_0)} \le C_2\) para algún \(C_2 \in \mathbb{R}\).

La misión entonces es acotar \(\norm{\exp\left((t-t_0) Q(t_0)\right)}\) pues en ese
caso tenemos que
\[
    \int_{0}^{\infty} \norm{e^{-t}B} dt
    \le
    \int_{0}^{\infty} e^{-t}\norm{B} dt
    < \infty
\]
y por teorema se concluiría que el sistema es estable.

Para acotar \(\norm{\exp\left((t-t_0) Q(t_0)\right)}\) necesitamos
caracterizarla un poco más. Dado que \(\sin(t) A \sin(t_0) A = \sin(t_0) A \sin(t)
A\), el propagador de nuestro sistema está dado por la fórmula
\[
    \Pi(t,t_0)
    =
    \exp\left(\int_{t_0}^{t} \sin(s) A ds \right)
    =
    \exp\left((\cos(t_0) - \cos(t)) A\right)
    =
    e^{-\cos(t)A} e^{\cos(t_0) A}
\]
y por lo tanto
\[
    \norm{\Pi(t,t_0)} \le e^{2\norm{A}} < \infty
\]
así que comparando con la forma que nos da Floquet deducimos que
\[
    \norm{e^{(t-t_0) Q(t_0)}} < \infty
.\]

% P2
\begin{problema}
    Considere el sistema \(\dot{x} = (A+B(t))x\) donde
    \[
        A =
        \begin{pmatrix}
        -4 & 2\\
        3 & -2
        \end{pmatrix}
        \hspace{1cm}y\hspace{1cm}
        B(t) =
        \begin{pmatrix}
        e^{-t^2} & \epsilon \cos(t)\\
        \epsilon \sin(t) & -1/(t^2+1)
        \end{pmatrix}
        \quad, \epsilon \in \mathbb{R}
    .\]
    Demuestre que existe \(\epsilon_0>0\), tal que si \(\abs{\epsilon} <
    \epsilon_0\), todas las soluciones del sistema \(x(t) \to 0\) cuando \(t\to
    \infty\).
\end{problema}
% R2
Consideremos el sistema no pertubado \(\dot{x} = Ax\).
El propagador asociado este sistema es
\[
	\Pi_A(t,t_0) = \exp\left((t-t_0) A\right)
.\]
Notar que los valores propios de \(A\) son \(z_{\pm} = -3 \pm \sqrt{7}\) ambos
reales negativos, así que las soluciones a este sistema son asintóticamente
estables.

Consideremos el sistema perturbado
\[
	\dot{x} = Ax + \underbrace{\begin{pmatrix} e^{-t^2} & 0\\ 0 & -1/(t^2+1)
	\end{pmatrix}}_{D(t)}
.\]
Como \(\norm{D(t)} \xrightarrow{t\to \infty} 0\) y \(A\) es AE (y constante), el sistema
entero es AE. Además, existen constantes \(C, \alpha > 0\) tal que
\[
	\Pi_{A+D}(t,s) \le C e^{-\alpha(t-s)}
.\]
Finalmente, llamemos \(H(t) = B(t) - D(t)\). Sabemos que \(\norm{H(t)} <
\abs{\epsilon}\). Luego, para \(\abs{\epsilon} < \alpha/C \eqqcolon \epsilon_0\) se tiene que
\[
	\Pi_{A+B}(t,s) \le K e^{-(\alpha - \abs{\epsilon}C)(t-s)}
	\xrightarrow{t\to \infty} 0
.\]
Por lo tanto el sistema es AE.

% P3
\begin{problema}
    Suponga que {\hfill Teschl. Problem 3.44}
    \[
        \int_{0}^{\infty} \norm{A(t)} dt < \infty
    .\]
    Muestre que todas las soluciones de \(\dot{x}(t) = A(t)x(t)\) convergen a un
    límite, es decir, para \(x(t)\) solución se tiene que
    \[
        \lim_{t \to \infty} x(t) = x_{\infty}
    .\]

    (\textit{Hint:} Muestre que todas las soluciones están acotadas y use la
    ecuación integral correspondiente.)
\end{problema}
Sabemos que las soluciones a este sistema son de la forma
\[
	x(t) = \Pi(t,t_0) x_0 \quad t\ge t_0 \ge 0
.\]

Tomando norma y usando la hipótesis vemos que las soluciones son acotadas, en
efecto:
\[
	\abs{x(t)}
	\le
	\norm{\Pi(t,t_0)} \abs{x_0}
	\le
	e^{\int_{t_0}^{t} \norm{A(s)} ds} \abs{x_0}
	< \infty
.\]
LLamemos \(M\) a una cota superior.

Para ver la convergencia, escribamos la ecuación integral,
\[
	x(t) = x_0 + \int_{t_0}^{t} A(s)x(s) ds
,\]
y notemos que para \(t\) suficientemenete grande y \(s > t\) se tiene que:
\[
	\abs{x(s) - x(t)}
	=
	\abs{\int_{t}^{s} A(r) x(r) dr}
	\le
	M
	\int_{t}^{s} \norm{A(r)} dr
.\]
Dado que \(\int_{0}^{\infty} \norm{A} < \infty\), la `cola' debe ir a cero,
es decir, para todo \(\epsilon > 0\) existe un \(T\) tal que
\[
	\int_{T}^{\infty} \norm{A(r)} dr < \epsilon/M
.\]
Luego, si \(t > T\) se sigue que:
\[
	\abs{x(s) - x(t)}
	\le
	M \int_{t}^{s} \norm{A(r)} dr
	\le
	M \int_{T}^{\infty} \norm{A(r)} dr
	< \epsilon
\]
y por lo tanto la solución \(x(t)\) es convergente.

% P4
\begin{problema}
\begin{enumerate}[label=(\alph*)]
    \item
    Sea \(H\) un espacio de producto interno y \(\left\{ u_j \right\}_{j=0}^{\infty}
    \subset H\) un conjunto ortonormal
    (numerable). Para \(f\in H\), muestre que
    \[
        f_n = \sum_{j=0}^{n} \langle u_j, f \rangle u_j
    \]
    es una secuencia de Cauchy.

    \item
    Muestre que si \(A\) es acotado, entonces todo valor propio \(\alpha\) de
    \(A\) satisface \(\abs{\alpha} \le \norm{A}\).
\end{enumerate}
\end{problema}
\begin{enumerate}[label=(\alph*)]
	\item % 4.a
	Por Pitágoras y la desigualdad de Bessel:
	\[
		\norm{f_n}^2
		=
		\sum_{j=0}^{n} \abs{\langle u_j, f\rangle}^2
		\le
		\norm{f}^2
	.\]
	Así que la sucesión \(\norm{f_n}^2\) es convergente, pues es creciente y
	acotada. Luego
	\[
		\lim_{n\to \infty} \norm{f_n}^2
		=
		\lim_{n\to \infty} \sum_{j=0}^{n} \abs{\langle u_j, f\rangle}^2
		< \infty
	\]
	y para que la serie converja, es necesario que la cola se vaya a cero. Así
	para \(\epsilon > 0\) existe un \(N\) tal que
	\[
		\sum_{j=N+1}^{M} \abs{\langle u_j, f\rangle}^2 < \varepsilon
		,\quad \forall M \ge N
	\]
	es decir,
	\[
		\norm{f_M - f_N}^2 < \varepsilon
	\]
	obteniéndose lo pedido.

	\item % 4.b
	Sea \(\alpha\) un valor propio de \(A\), tomemos \(f\) un vector propio
	asociado de norma \(1\). Luego
	\[
		\norm{A}
		=
		\sup_{\norm{g} = 1} \norm{Ag}
		\ge
		\norm{Af}
		=
		\norm{\alpha f}
		=
		\abs{\alpha}
	.\]
	Donde en la primera igualdad usamos que \(A\) es acotado.
\end{enumerate}


% P5
\begin{problema}
    Sea \(H\) un espacio de producto interno. Muestre que el conjunto de
    funciones infinitamente diferenciables con soporte compacto
    \(\mathcal{C}^{\infty}_{c}\left((a,b), \mathbb{C}\right)\) es denso en
    \(H\).

    (\textit{Hint:} Reemplace P(x) en la demostración del lema 5.8 por
    \(\int_{0}^{x} \exp((y(y-1))^{-1}) dy/\int_{0}^{1} \exp((y(y-1))^{-1}) dy\))
\end{problema}

% P6
\begin{problema}
    Sean \(\alpha, \beta\in \mathbb{R}\), \(\alpha > 0\). Considere el PVF
    \[
        \begin{cases}
            -u'' + \alpha u' = f(x) &, x\in (0,1)\\
            u + \beta u' = 0 &, x = 0, 1
        \end{cases}
    ,\]
    donde \(f\in \mathcal{C}[0,1]\).
    \begin{enumerate}[label=(\alph*)]
        \item
        Pruebe que existe una única solución del problema si y solo si
        \(\alpha \beta \ne -1\).

        \item
        Para \(\alpha \beta \ne 1\), determine la \textit{función de Green}
        correspondiente, i.e. la función \(G\colon [0,1]^2 \to \mathbb{R}\) tal
        que para cualquier \(f\in \mathcal{C}[0,1]\) la solución del problema es
        \[
            u(x) = \int_{0}^{1} G(x,t) f(t) dt
        .\]
    \end{enumerate}
\end{problema}
\begin{enumerate}
	\item % 6.a
	Transformemos el problema en uno de Sturm-Liouville. Multiplicando por el
	factor integrante \(e^{-\alpha x}\) obtenemos que la EDO es equivalente a
	resolver el sistema:
	\[
		Lu = (e^{-\alpha x} u')' = e^{-\alpha x} f(x)
	\]
	con dominio las condiciones de frontera \(B_0 u = B_1 u = u + \beta u' = 0\).

	Para que tenga solución única, \(z=0\) no debe ser valor propio, es decir
	\(Lu \ne 0\). Veamos cuándo se cumple esto.
	\[
		Lu = 0 \iff -u'' + \alpha u' = 0 \iff u(x) = c_1 + c_2 e^{-\alpha x}
	.\]
	Mirando las condiciones de frontera, tenemos que
	\[
	\begin{cases}
		x=0 & c_1 + c_2 - \alpha \beta c_2 = 0\\
		x=1 & c_1 + c_2 e^{-\alpha} - \alpha \beta c_2 e^{-\alpha} = 0
	\end{cases}
	.\]
	Despejando \(1-\alpha\beta\) de la primera ecuación y poniéndola en la
	segunda llegamos que \(c_1 (1-e^{-\alpha}) = 0\). Dado que \(\alpha > 0\),
	\(c_1 = 0\), se sigue que \(c_2 (1-\alpha\beta) = 0\), así que para \(c_2 =
	0\) o \(\alpha\beta = 1\) se cumple la igualdad.

	Concluimos entonces, que \(\alpha\beta \ne 1\) para que \(z = 0\) no sea
	valor propio. En tal caso la solución de \(Lu = e^{-\alpha x} f(x)\) es
	única.

	\item % 6.b
	Para \(\alpha\beta \ne 1\), el operador \(L\) es invertible y está dado por
	la fórmula:
	\[
		u(x)
		=
		\frac{1}{W} \left(
			u^{1}(x) \int_{0}^{x} u^{0}(t) e^{-\alpha t} f(t) dt
			+
			u^{0}(x) \int_{x}^{1} u^{1}(t) e^{-\alpha t} f(t) dt
		\right)
		=
		\int_{0}^{1} G(x,t) f(t) dt
	\]
	donde \(-W G(x,t)\) vale \(u^{1}(x) u^{0}(t)\) para \(x \le t\) y \(u^{0}(x)
	u^{1}(t)\) para \(x\ge t\).

	Tomemos \(u^{0}(x) = -(1-\alpha\beta) + e^{-\alpha x}\) y \(u^1(x) =
	-(1-\alpha \beta) + e^{\alpha (1-x)}\). Calculemos \(W\).
	\[
		W = \abs{
		\begin{matrix}
			-(1-\alpha\beta) + e^{\alpha(1-x)} & -(1-\alpha\beta) + e^{-\alpha x}\\
			-\alpha e^{\alpha(1-2x)} & -\alpha e^{-2\alpha x}
		\end{matrix}
		}
		\overset{x=0}{=}
		\alpha (1-e^{\alpha})
	.\]
	De esta forma,
	\[
		G(x,t)
		=
		-\frac{1}{\alpha(1-e^{\alpha})}
		\begin{cases}
		{\left(\alpha \beta\alpha + e^{\left(\alpha t\right)} - 1\right)}
		{\left(\alpha \beta\alpha + e^{\left(-\alpha {\left(x - 1\right)}\right)}
		- 1\right)} & x \le t\\
		{\left(\alpha \beta + e^{\left(\alpha x\right)} - 1\right)}
		{\left(\alpha \beta + e^{\left(-\alpha {\left(t - 1\right)}\right)} -
		1\right)} & x \ge t
		\end{cases}
	.\]
\end{enumerate}

% P7
\begin{problema}
    Encuentre los valores propios y las funciones propias del operador de
    Sturm-Liouville definido por
    \[
        Ly = -y'',\quad
        \dom L \coloneqq \left\{ u\in \mathcal{C}^2[0,1]\colon u(0)=u'(1)=0 \right\}
    .\]
\end{problema}
Tenemos que resolver la ecuación \(-y'' - \lambda y = 0\). Las soluciones a este
sistema están determinadas por la ecuación característica \(-z^2 - \lambda =
0\). Veamos los distintos casos:
\begin{itemize}
	\item Caso \(\lambda = 0\): \(y\) es una función lineal y las condiciones de
	frontera nos dicen que \(y\equiv 0\).

	\item Caso \(\lambda < 0\): \(z = \pm \sqrt{\abs{\lambda}}\) y las
	soluciones son de la forma \(y(x) = c_1 e^{x z_{+}} + c_2 e^{x z_{-}}\).
	Aplicando las condiciones de frontera se concluye que \(y \equiv 0\).

	\item Caso \(\lambda > 0\):
	\(z = \pm \sqrt{\lambda} i\) y las soluciones son de la forma
	\(y(x) = c_1 \cos(x\sqrt{\lambda}) + c_2 \sin(x\sqrt{\lambda})\).
	Aplicando las condiciones de frontera, nos da que
	\[
	\begin{cases}
		0 = y(0) = c_1 \cos(0) + c_2 \sin(0)
			&\implies c_1 = 0\\
		0 = y'(1) = c_2 \cos(\sqrt{\lambda}) \sqrt{\lambda}
			&\implies c_2 = 0 \lor \sqrt{\lambda} = \frac{\pi}{2} + k\pi
	\end{cases}
	\]
	donde \(k\in \mathbb{N}\). Luego, los valores propios son \(\lambda_{k} =
	(\pi/2 + k\pi)^2\) y una función propia asociada es de la forma \(y_k(x) = c_2
	\sin(\sqrt{\lambda_k} x)\)
\end{itemize}

% P8
\begin{problema}
\begin{enumerate}
	\item Demuestre la {\it Desigualdad de Sobolev}
	\[
		\int_{0}^{\pi} \abs{u(x)}^2 dx
		\le
		\int_{0}^{\pi} \abs{u'(x)}^2 dx
	\]
	Para todo \(u\in \mathcal{C}^{1}[0,\pi]\) con \(u(0) = u(\pi) = 0\).

	\textit{Sugerencia:} Utilice el Principio de Rayleigh-Ritz para el operador
	\(L_0 = -d^2/dx^2\) en \([0,\pi]\) con condiciones homogéneas de Dirichlet.

	\item Suponga que \(q\in \mathcal{C}[0,\pi]\) satisface \(\min_{[0,\pi]} q >
	-1\). Pruebe que los valores propios del operador de Sturm-Liouville
	\[
		Ly
		=
		-\frac{d^2 y}{dx^2} + q(x)y
	,\]
	con dominio \(\mathcal{D}(L) \coloneqq \left\{ u\in \mathcal{C}^2[0,\pi], u(0)
	= u(\pi) = 0 \right\}\) son positivos.
\end{enumerate}
\end{problema}
\begin{enumerate}
	\item % 8.1
	En primer lugar notemos que
	\[
		\langle u, u\rangle^2 = \int_{0}^{\pi} \abs{u(x)}^2 dx
		\text{ y }
		\langle u, -u''\rangle^2 = \int_{0}^{\pi} \abs{u'(x)}^2 dx
	.\]
	Así que debemos probar que \(\langle u, u\rangle \le \langle u,
	-u''\rangle\).

	Consideremos el operador \(L = -\frac{d^2}{dx^2}\) con dominio los \(u\) tal
	que \(u(0) = u(\pi) = 0\). Estudiemos sus valores
	propios, es decir, la ecuación \(-y'' - \lambda y = 0\) cuyas soluciones
	están determinadas por las raíces de la ecuación \(-z^2 - \lambda = 0\).
	\begin{itemize}
		\item Caso \(\lambda = 0\colon\) Las soluciones son lineales y las condiciones de
		frontera hacen que la única que cumpla sea la función nula.

		\item Caso \(\lambda < 0\colon\) Las soluciones de la ecuación son \(z_{\pm}
		= \pm\sqrt{\abs{\lambda}}\) y por lo tanto \(y\) es de la forma \(y(x) =
		c_1 e^{x z_{-}} + c_2 e^{x z_{+}}\). Aplicando las condiciones de
		frontera:
		\[
			\begin{cases}
				x=0&, 0 = y(0) = c_1 + c_2\\
				x=\pi&, 0 = y(\pi) = c_1 e^{\pi z_{-}} + c_2 e^{\pi z_{+}}
			\end{cases}
		.\]
		Vemos que no hay soluciones no nulas (\(c_1 (e^{-\pi \sqrt{\lambda}} -
		e^{\pi \sqrt{\lambda}}) = 0 \iff c_1 = 0\)).

		\item Caso \(\lambda > 0\colon\) \(z_{\pm} = \pm \sqrt{\lambda} i\) y las
		soluciones son de la forma \(y(x) = c_1 \cos(x \sqrt{\lambda}) + c_2
		\sin(x \sqrt{\lambda})\). Aplicando las condiciones de frontera:
		\[
			\begin{cases}
				x=0&, 0 = y(0) = c_1\\
				x=\pi&, 0 = y(\pi) = c_2 \sin(\pi \sqrt{\lambda})
			\end{cases}
		.\]
		Aquí \(\lambda_{k} = k^2\), y las funciones propias asociadas a
		\(\lambda_k\) son de la forma
		\(y_k(x) = c_{2,k} \sin\left(x k\right)\)
	\end{itemize}

	Por el principio de Rayleigh se cumple que
	\[
		1
		=
		\inf_{u\in \mathcal{D}(L)}
		\frac{\langle u, Lu\rangle}{\langle u, u\rangle}
		\implies
		\langle u, u\rangle \le \langle u, Lu\rangle
	.\]
	Que es lo que queríamos probar.

	\item % 8.2
	Sea \(f \in \mathcal{D}(L)\) de norma \(1\). Luego,
	\begin{align*}
		\langle f, Lf\rangle
		&=
		\int_{0}^{\pi} \overline{f(x)}\left(-f''(x) + q(x) f(x)\right) dx
		\\&=
		\underbrace{-\overline{f(x)}f'(x)\Big\vert_{0}^{\pi}}_{0}
		+
		\int_{0}^{\pi} \overline{f'(x)} f'(x) dx
		+
		\int_{0}^{\pi} \overline{f(x)} f(x) q(x) dx
		\\&=
		\int_{0}^{\pi} \abs{f'(x)}^2 dx
		+
		\int_{0}^{\pi} \abs{f(x)}^2 q(x) dx
		\\&\overset{1.}{\ge}
		\int_{0}^{\pi} \abs{f(x)}^2 (1+q(x)) dx
		\\&=
		1 + \min_{x\in \left[0,\pi\right]} q(x) > 0
	\end{align*}
	Por el principio de Rayleigh los valores propios son positivos.
\end{enumerate}
\end{document}

