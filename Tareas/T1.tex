\documentclass[10pt]{article}

\usepackage[margin=.7in]{geometry}
\usepackage{enumitem}
\usepackage{multicol}
\usepackage{float}
\usepackage{multicol}
\usepackage{tikz,pgfplots}
\usepackage{amsmath,amsthm,amssymb}
\usepackage{mathtools}
\usepackage[utf8]{inputenc}
\usepackage[spanish]{babel}
\usetikzlibrary{babel}

\usepackage{mdframed}

\usepackage{amstext} % for \text macro
\usepackage{array}   % for \newcolumntype macro
\newcolumntype{C}{>{$}c<{$}} % math-mode version of "c" column type
\newcolumntype{L}{>{$}l<{$}} % math-mode version of "l" column type

\newcounter{problemctr}[section]
\newenvironment{problema}{
    \refstepcounter{problemctr}
    \mdfsetup{%
        frametitle={\tikz\node[fill=white,rectangle,inner sep=1pt,outer
        sep=1pt]{Problema \theproblemctr};},
        frametitleaboveskip=-0.5\ht\strutbox,
        frametitlealignment=\center
    }%
    \begin{mdframed}[style=exampledefault]
    }{\end{mdframed}
}

\newlist{plist}{enumerate}{1}
\setlist[plist]{label=(\alph*),leftmargin=2em,rightmargin=2em,itemindent=2pt,itemsep=0pt,topsep=0pt}

\newlist{clist}{itemize}{1}
\setlist[clist]{label=\(\bullet\), listparindent=1em, leftmargin=2pt,itemindent=2pt}

\usepackage[T1]{fontenc}

%\usepackage{subfiles}

% Resize abs and norm
\DeclarePairedDelimiter{\abs}{\lvert}{\rvert}
\DeclarePairedDelimiter{\norm}{\|}{\|}
\makeatletter
\let\oldabs\abs
\def\abs{\@ifstar{\oldabs}{\oldabs*}}
\let\oldnorm\norm
\def\norm{\@ifstar{\oldnorm}{\oldnorm*}}
\makeatother

\newcommand\sideToSide[3]%
{
    \begin{minipage}{#1\textwidth}
       #2
    \end{minipage}
    \begin{minipage}{\dimexpr\textwidth-#1\textwidth}
       #3
    \end{minipage}
}

%\begin{document}
%    \subfile{Ayudantias/A5}
%    \subfile{Ayudantias/A10}
%\end{document}



\usetikzlibrary{intersections}
\begin{document}

{\sc Tarea 1 - EDO\hfill Sebastián Sánchez}

\begin{center}
Colaboladores:
Felipe González,
Claudio Meléndez,
José Cuevas,
Sebastián Lepe,
Pablo Navarro,
Mathias Luengo,
Camila Guajardo Vásquez.
\end{center}

% PROBLEMA 1
\begin{problema}
Resuelva las siguientes EDO's:
\begin{multicols}{3}
\begin{enumerate}[label=(\alph*)]
\item \(\dot{x} = x(1-x)\).
\item \(\dot{x} = x\sin(t)\).
\item \(\dot{x} = \frac{3x-2t}{t}\).
\item \(y' = y^2 - \frac{y}{x} - \frac{1}{x^2}\).
\item \(y' = \frac{y}{x} - \tan \left(\frac{y}{x}\right)\).
\end{enumerate}
\end{multicols}
\end{problema}
\begin{proof}[Solución]\ \\
\begin{plist}
\item \(\dot{x} = x(1-x)\).

La ecuación es lineal y autónoma de primer orden.
Consideremos las diferentes condiciones iniciales \(x(0) = x_0\).
\begin{clist}
    \item Si \(x_0 = 0\) la derivada de la solución es cero y por lo tanto
    concluimos que \(\varphi \equiv 0\) es la única solución.

    \item Si \(x_0 = 1\), \(\dot{x}\) es nula y por lo tanto la solución
    \(\varphi(t)\) es constante y vale \(1\).

    \item Si \(0 < x_0 < 1\) se tiene que \(0 < \dot{x}(0) < 1\), por lo tanto la
    solución \(\varphi(t)\) es creciente. Los valores de \(x\) donde \(\dot{x}\)
    no se anula van desde van desde \(x_1 = 0\) hasta \(x_1 = 1\).
    Despejando la ecuación e integrando obtenemos la inversa de la solución
    \[
        F(x;x_0)
        =
        \int_{x_0}^{x} \frac{du}{u\left(1-u\right)}
        =
        \int_{x_0}^{x} \frac{1 \pm u}{u\left(1-u\right)} du
        =
        \log \left(\frac{x}{1-x}\right)
        -
        \log \left(\frac{x_0}{1-x_0}\right)
    .\]
    Luego, el intervalo máximo de la solución está dado por
    \[
        T_{+} = \lim_{x\uparrow 1} F(x;x_0) = \infty
        \hspace{1.5cm}
        T_{-} = \lim_{x\downarrow 0} F(x;x_0) = -\infty
    .\]
    Además, podemos calcular la inversa explícitamente. Así, la solución
    \(x(t)\) es
    \[
        x(t) = \frac{e^{t}x_0}{1 - x_0 + e^{t}x_0}.
    \]

    \item Si \(x_0 > 1\) entonces para \(x > 1\) la derivada \(\dot{x}(x) < 0\) y
    por lo tanto esperamos una solución decreciente. Dado que en \(x = 1\) la
    derivada se anula, deducimos que (de existir) una solución se debe ver como
    una función decreciente con asíntota en \(x = 1\). De lo anterior también
    deducimos que el intervalo máximo donde \(x(1-x)\) no se desvanece va desde
    \(x_1 = 1\) y \(x_2 = \infty\).

    Como antes, podemos computar la inversa de la solución, dandonos la
    expresión
    \[
        F(x;x_0)
        =
        \int_{x_0}^{x} \frac{du}{u-u^2}
        =
        \log\left(\frac{x}{x-1}\right) - \log\left(\frac{x_0}{x_0-1}\right)
    .\]
    Dado que \(1/(u-u^2)\) es negativa en \(\left(x_1,x_2\right)\) la función
    \(F(x;x_0)\) es monótona decreciente. Luego, para obtener el intervalo
    máximo de la solución basta evaluar en los límites.
    \[
        T_{-} = \lim_{x\uparrow \infty} F(x;x_0) = -\log\left(\frac{x_0}{x_0-1}\right)
        \hspace{1.5cm}
        T_{+} = \lim_{x\downarrow 1} F(x;x_0) = +\infty
    .\]
    Despejando la ecuación explícita obtenemos la misma que antes.
    \[
        x = \frac{e^{t}x_0}{1 - x_0 + e^{t}x_0}.
    \]

    \item Si \(x_0 < 0\), la derivada es negativa por lo que esperamos una
    solución decreciente. Los valores dónde \(\dot{x}\) no se anula va desde
    \(x_1 = -\infty\) hasta \(x_2 = 0\). Dado que la derivada decrece a medida
    que nos acercamos a \(x=0\), esperamos que la solución tenga una asíntota en
    este valor (a medida que retrocedemos en el tiempo).

    Como antes, la inversa de la solución nos da que
    \[
        F(x;x_0)
        =
        \int_{x_0}^{x} \frac{du}{u(1-u)}
        =
        \log\left(\frac{x}{x-1}\right) - \log\left(\frac{x_0}{x_0 - 1}\right)
    .\]
    Dado que tenemos argumentos negativos, vemos que \(F(x;x_0)\) es
    decreciente y monótona. Luego, los intervalos máximos son
    \[
        T_{-} = \lim_{x\uparrow 0} F(x;x_0)
        = -\infty
        \hspace{1.5cm}
        T_{+} = \lim_{x\downarrow -\infty} F(x;x_0)
        = -\log\left(\frac{x_0}{x_0-1}\right)
    .\]
    La solución explota en tiempo finito en el futuro. La fórmula explícita es
    idéntica a las anteriores.
\end{clist}
\item \(\dot{x} = x\sin(t)\).

La ecuación es de primer orden y separable. Definamos \(x_0 = x(t_0)\).

Veamos las diferentes condiciones iniciales. Dado que la función seno es
periódica, esperamos soluciones periodicas.

\begin{clist}
    \item Si \(x_0 = 0\), entonces una solución es \(x\equiv 0\) y vale
    para todo \(t\in \mathbb{R}\).

    \item Si \(x_0 > 0\). El intervalo máximo dónde \(x\) no se anula va de
    \(x_1=0\) hasta \(x_2 = \infty\). Despejando e integrando tenemos que
    la inversa de la solución es
    \[
        F(x;x_0) = \int_{x_0}^{x} \frac{du}{u} = \log(x) - \log(x_0)
        =
        \cos(t_0) - \cos(t)
    .\]
    Donde \(x \in (0, \infty)\). Así, el intervalo máximo de la solución es
    \[
        T_{+} = \lim_{x\uparrow \infty} F(x;x_0) = \infty
        \hspace{1.5cm}
        T_{-} = \lim_{x\downarrow 0} F(x;x_0) = -\infty
    .\]
    Y la podemos calcular explícitamente, obteniendo
    \[
        x(t) = x_0 \exp\left(\cos(t_0) - \cos(t)\right)
    .\]

    \item Si \(x_0 < 0\) el procedimiento es similar y de hecho nos da la misma
    expresión y los mismos tiempos.
\end{clist}

\item \(\dot{x} = \frac{3x-2t}{t} = 3\frac{x}{t}-2\).

Es una ecuación de primer orden no homogénea. Notar que \(t \ne 0\).
Tomemos \(y = \frac{x}{t}\), de esta forma nos queda el sistema separable
\[
    \dot{y}
    =
    \frac{\dot{x}}{t} - \frac{x}{t^2}
    =
    \frac{2}{t} \left(y - 1\right)
.\]

Ahora debemos analizar los distintos PVI. Llamemos \(y_0 \coloneqq y(t_0) =
x(t_0)/t_0\).

\begin{clist}
    \item Si \(y_0 = 1\), una solución es la función constante \(y\equiv 1\).

    \item Si \(y_0 > 1\), entonces la imagen de la solución va desde \(y_1 = 1\) a
    \(y_2 = \infty\). Dado que \(t\ne 0\), debemos analizar los casos de \(t\)
    positivo y negativo. Sin perjuicio de lo enterior, trataremos de posponerlo
    hasta que sea necesario, pues por la forma
    de la ecuación se puede decir que hay una simetría `refleja' con respecto al
    tiempo.

    Despejando la ecuación de la manera usual para obtener la inversa.
    \[
        F(y;y_0) = \int_{y_0}^{y} \frac{du}{u-1} = \log(y-1) - \log(y_0 - 1)
    .\]
    Así, el intervalo máximo de la solución está dado por
    \[
        T_{+} = \lim_{y\uparrow \infty} F(y;y_0) = \infty\\
        \hspace{1cm}
        T_{-} = \lim_{y\downarrow 1} F(y;y_0) = -\infty
    .\]
    Además, podemos computar explicitamente la solución donde se aprecia la
    simetría con respecto a \(t\).
    \begin{align*}
        \log(y-1) - \log(y_0 - 1) &= \log(t^2) - \log(t_0^2)
        \\\Rightarrow
        y(t) &= (y_0 - 1) \frac{t^2}{t_0^2} + 1
    \end{align*}
\end{clist}

\item \(y' = y^2 - \frac{y}{x} - \frac{1}{x^2}\).

La ecuación es de Ricatti, por lo que necesitamos saber una solución particular
\(y_p(x)\). Una inspección arroja que \(y_p = 1/x\) funciona. Luego, con el
cambio de variable \(u = 1/(y-y_p)\) nos queda el sistema lineal equivalente
\[
    \dot{u} = -\frac{1}{x} u - 1 \tag{\(\star\)}
.\]
Usaremos el método del propagrador. Tenemos que
\[
    P(v;w)
    =
    \exp\left(\int_{w}^{v} -\frac{dz}{z} \right)
    =
    \exp\left(\log(w/v)\right)
    =
    \frac{w}{v}
.\]
Por lo tanto la solución es
\[
    u(x)
    =
    u_0 P(x;x_0) + \int_{x_0}^{x} -P(x;s) ds
    =
    u_0 \frac{x_0}{x} - \frac{1}{x} \int_{x_0}^{x} s ds
    =
    u_0 \frac{x_0}{x} - \frac{1}{2x} \left(x^2 - x_0^2\right)
.\]
Del sistema original ya vemos que \(x_0,x \ne 0\) así que el propagador no se
anula. Por el cambio de variables es claro que \(u \ne 0\) y en particular
\(u_0 \ne 0\). Así, \(x_0 \in (0, \pm \infty\) y \(u_0 \in (0, \pm \infty\).

\item \(y' = \frac{y}{x} - \tan \left(\frac{y}{x}\right)\).

Definiendo \(w = \frac{y}{x}\) nos queda la ecuación separable:
\[
    \dot{w}
    =
    \frac{\dot{y}}{x} + \frac{y}{x^2}
    =
    \frac{w-\tan(w)}{x} + \frac{w}{x}
    =
    -\frac{\tan(w)}{x}
.\]
Como tangente se anula en \(k\pi\) y se indefine en \(k\pi \pm \frac{\pi}{2}\) (con
\(k\in \mathbb{Z}\)). Debemos analizar estos distintos casos.
\begin{clist}
    \item Si \(w_0 = w(x_0) = y(x_0)/x_0 = 0\), entonces la función constante
    \(w \equiv 0\) es una solución.

    \item Si \(w_0 \in \left(k\pi, k\pi + \frac{\pi}{2}\right)\). Entonces el intervalo máximo dónde tangente no se
    anula va desde \(w_1 = k\pi\) hasta \(w_2 = k\pi + \frac{\pi}{2}\).

    Despejando e integrando, la inversa de la solución nos da
    \[
        F(w;w_0)
        =
        \int_{w_0}^{w} \frac{du}{\tan(u)}
        =
        \log(\abs{\sin(w)}) -
        \log(\abs{\sin(w_0)})
    .\]
    Notar que sin importar la paridad de \(k\), la tangente es positiva
    y por lo tanto la integral es creciente. Luego, como el dominio de la solución
    es la imagen de la inversa, basta evaluar en los extremos. De esta forma el
    tiempo máximo y mínimo respectivamente son
    \[
        T_{+} = \lim_{w \uparrow k\pi + \frac{\pi}{2}} F(w;w_0) =
        -\log(\abs{\sin(w_0)})
        \hspace{1.5cm}
        T_{-} = \lim_{w \downarrow k\pi} F(w;w_0) = -\infty
    .\]
    Finalmente, despejando la siguiente expresión (que salió de haber integrado)
    \[
        \log(\abs{\sin(w)}) - \log(\abs{\sin(w_0)})
        =
        \log(\abs{x_0}) - \log(\abs{x})
    .\]
    Podemos encontrar la ecuación explícita de la solución (en términos de \(w\))
    \[
        w = \arcsin\left(\frac{x_0}{x} \sin(w_0)\right)
    .\]
    Deshaciendo el cambio, la solución en términos de \(y\) es
    \[
        y = x\arcsin\left(\frac{x_0}{x} \sin(y_0/x_0) \right)
    .\]

    \item Si \(w\in \left(k\pi - \frac{\pi}{2}, k\pi\right)\) la tangente es
    negativa y por lo tanto \(F(w;w_0)\) es decreciente. Luego, lo único que
    cambia es el intervalo máximo de la solución, siendo este delimitado por
    \[
        T_{-} = \lim_{w\uparrow k\pi} F(w;w_0) = -\infty
        \hspace{1.5cm}
        T_{+} = \lim_{w\downarrow k\pi + \frac{\pi}{2}} F(w;w_0) =
        \log(\abs{\sin(w_0)})
    .\]
\end{clist}
\end{plist}
\end{proof}

% PROBLEMA 2
\begin{problema}
Investigue la unicidad de la ecuación diferencial
\[
    \dot{x}
    =
    \begin{cases}
        -t\sqrt{\abs{x}} &, x \ge 0\\
        t\sqrt{\abs{x}} &, x \le 0
    \end{cases}
.\]
Muestre que el PVI \(x(0) = x_0\) tiene una única solución global para cada \(x_0
\in \mathbb{R}\). Sin embargo, muestre que las soluciones globales de todas
formas se intersectan.

\textit{Hint: Note que si \(x(t)\) es una solución, también lo es \(x(-t)\) y
\(-x(t)\). Así que solo hace falta considerar los casos \(x_0 \ge 0\) y \(t_0
\ge 0\)}.
\end{problema}
\begin{proof}
La ecuación es separable. Analicemos las condiciones iniciales.

\begin{clist}
    \item Si \(x_0 = 0\) entonces \(x(t) = -x(t) = x(-t) \equiv 0\) es una solución
    global.

    \item Si \(x_0 > 0\) el IM es \((x_1, x_2) = (0,\infty)\). Despejando e
    integrando de la manera usual tenemos que la solución es
    \[
        x(t) = \left(\sqrt{x_0} + \frac{t_0^2 - t^2}{4}\right)^2
        \hspace{1cm}
        -2\sqrt{x_0} < t < \infty
    .\]
    Dado que las apariciones del tiempo son cuadráticas identificamos la
    simetría \(t \to -t\), así que \(x(-t)\) también es solución. Además,
    la solución es única puesto que \(F(x;x_0) =
    \int_{x_0}^{x} \frac{du}{\sqrt{u}}\) es estrictamente creciente en \((x_1, x_2)\).

    \item Si \(x_0 < 0\) el IM es \((x_1, x_2) = (-\infty, 0)\). Luego, la
    solución es
    \[
        x(t) = -\left(\sqrt{-x_0} + \frac{t_0^2 - t^2}{4}\right)^2
        \hspace{1cm}
        -\infty < t < 2\sqrt{-x_0}
    .\]
    Dado que \(-x_0 > 0\) tenemos la misma solución que antes pero reflejada con
    respecto al eje \(t\) (es decir, \(-x(t)\) también es solución).
    Por otro lado, podemos observar la misma simetría con respecto a \(t\).
    Finalmente, la solución es única dado que \(F(x;x_0) = \int_{x_0}^{x} \frac{du}
    {\sqrt{-u}}\) es estrictamente creciente en \((x_1, x_2)\).
\end{clist}

Notadas las simetrías, seguiremos en análisis considerando \(x_0, t_0 \ge 0\).
Como la solución una cuártica con múltiplicidad 2 y las raíces existen
(podemos despejar \(t\) explícitamente) hay dos intersecciones con la solución
\(x\equiv 0\), ahora, bajo las restricciones de \(t\) sólo una es parte
de la solución particular para ese \(x_0\) y en nuestro análisis es la raíz
positiva.

Para ver que intersecta con otras soluciones no nulas, consideremos
\(x_0\) fijo. Para un tiempo inicial \(t_a\), el área de interés es el trozo que va
desde \(-2\sqrt{x_0}\) hasta la primera raíz positiva de la solución (llamemosla
\(T_a\)) ya que aquí podrían intersectarse. Y en efecto, si \(a < b\) entonces
\(T_a < T_b\), y por lo tanto la curva que sube desde el tiempo \(T_a\) debe
intersectar a la curva que baja y que toca el \(0\) en el tiempo \(T_b\).
\begin{center}
\begin{tikzpicture}[scale=3]
    \def\x{1};
    \def\a{0};
    \def\b{2};
    \def\l{1.1};
    \clip (0,1.7) rectangle (4,-.5);
    \draw[->] (\l, 0) -- (3, 0) node[near end,below] {\(t\)};
    \draw[domain=\l:3, smooth, variable=\t]
        plot ({\t}, {(sqrt(\x) + ((\a)^2 - (\t)^2)/4)^2}) node[below right]
        {\(x_a\)};
    \draw[domain=\l:3, smooth, variable=\t, dashed]
        plot ({\t}, {(sqrt(\x) + ((\b)^2 - (\t)^2)/4)^2})
        node[above left] {\(x_b\)};
\end{tikzpicture}
\end{center}
\end{proof}

% PROBLEMA 3
\begin{problema}
Considere la ecuación
\[
    F(x,y) = 0, \qquad F \in C^2(\mathbb{R}^2, \mathbb{R}^2)
.\]
Suponga que \(y(x)\) soluciona esta ecuación. Muestre que \(y(x)\) satisface
\begin{equation*}\label{P3:eq1}
    p(x,y) y' + q(x,y) = 0 \tag{\(\star\)}
\end{equation*}
Donde
\[
    p(x,y) = \frac{\partial F(x,y)}{\partial y}
    \qquad
    q(x,y) = \frac{\partial F(x,y)}{\partial x}
.\]
Muestre que tenemos
\[
    \frac{\partial p(x,y)}{\partial x}
    =
    \frac{\partial q(x,y)}{\partial y}
.\]

Para el recíproco, una ecuación diferencial de primer orden como la
de arriba~\eqref{P3:eq1} (con coeficientes arbitrareos \(p(x,y)\) y \(q(x,y)\))
que satisface la
última ecuación es llamada \textbf{exacta}. Muestre que si la ecuación es
exacta, entonces existe una función \(F\) (como la de arriba) correspondiente.
Encuentre una fórmula explícita para \(F\) en términos de \(p\) y \(q\). ¿Está
\(F\) determinada de manera única por \(p\) y \(q\)?

Muestre que
\[
    (4bxy + 3x + 5)y' + 3x^2 + 8ax + 2by^2 + 3y = 0
\]
es exacta. Encuentre \(F\) y las soluciones.
\end{problema}
\begin{proof}
Supongamos que \(y(x)\) es solución del sistema. Dado que \(F(x,y) \equiv 0\), la
derivada también debe ser nula. Luego, por la regla de la cadena tenemos que
\[
    0
    =
    d_x F(x,y(x))
    =
    \partial_x F(x,y) + \partial_y F(x,y) y'
    =
    q(x,y) + p(x,y) y'
.\]

Como \(F\) es de clase \(C^2\), el teorema de Clairaut nos dice que el orden de las
segundas derivadas es intercambiable. El resultado es directo pues
\[
    \partial_y q(x,y) = \partial_y \partial_x F(x,y)
    =
    \partial_x \partial_y F(x,y) = \partial_x p(x,y)
.\]

Ahora veamos el converso. Supongamos que tenemos funciones \(p(x,y), q(x,y)\) que
satisfacen
\[
    \partial_y q(x,y) = \partial_x p(x,y)
    \hspace{.5cm} \text{y} \hspace{.5cm}
    p(x,y)y' + q(x,y) = 0
.\]

Reexpresando la segunda ecuación como
\[
    p(x,y) dy + q(x,y) dx = 0
\]
e integrando sobre un camino cerrado con interior (\(C^{\circ}\)) acotado y
simplemente conexo (muchas hipótesis, pero como tenemos todo \(\mathbb{R}^2\)
para elegir, le damos nomás) tenemos que (por Green)
\[
    \int_{C} p(x,y) dy + q(x,y) dx
    =
    \int_{C^{\circ}} \partial_x p(x,y) - \partial_y q(x,y) dx dy
    =
    0
.\]
Definamos entonces
\[
    F(x,y)
    =
    \int_{C} p(x,y) dy  + q(x,y) dx
.\]
De esta forma,
\[
    \partial_y F(x,y)
    =
.\]

Veamos que el sistema
\[
    \underbrace{(4bxy + 3x + 5)}_{p(x,y)}y'
    +
    \underbrace{3x^2 + 8ax + 2by^2 + 3y}_{q(x,y)}
    = 0
\]
es exacto. Basta comparar derivadas parciales
\[
    \partial_x p(x,y) = 4by + 3 = \partial_y q(x,y)
.\]
Así que es exacta. Luego,
\[
    F(x,y) = \int 4bxy + 3x + 5 dx = 2byx^2 + \frac{3}{2} x^2 + 5x + C(y)
.\]
Donde \(C(y)\) es una constante dependiendo de \(y\). Finalmente, las soluciones
están dadas por (usando que \(F(x,y) = 0\))
\[
    y(x) = -\frac{(3/2)x^2 + 5x - C(y)}{2bx^2}
.\]

\end{proof}

% PROBLEMA 4
\begin{problema}
Sean \(\tau > 0\) y \(\gamma > 0\) constantes. Considere el problema de valor
inicial
\[
    \dot{x} = \gamma \sqrt{\abs{x}} - \tau x
    ,\qquad x(0) = x_0
.\]
\begin{plist}
    \item Resuelva el problema. (\textit{Sugerencia}: La EDO es de tipo
    Bernoulli).

    \item Analice la unicidad de la solución y determine el intervalo máximo de
    definición. Si hay falla de unicidad, explique por qué esto no contradice el
    teorema de Picard-Lindelöf.

    \item Analice el comportamiento a largo plazo (\(t\to \infty\)) cuando \(x_0
    > 0\).
\end{plist}
\end{problema}
\begin{proof}\ \\
\begin{plist}
    \item En primer lugar notemos que el sistema es autónomo.

    Para \(x_0 = 0\), la función \(x\equiv 0\) es una solución.

    Para \(x_0 > 0\), la ecuación queda
    \[
       \dot{x} = \gamma x^{1/2} - \tau x
    .\]
    La cual es Bernoulli con \(n = 1/2\). Luego, el cambio de variables \(y =
    x^{1/2}\) nos da el sistema lineal
    \[
       \dot{y}
       =
       \frac{\gamma - \tau y}{2}
    .\]
    Con intervalo máximal en \(y\) de \(y_2 = \frac{\tau}{\gamma}\) y \(y_1
    = -\infty\).

    Usando el método del propagador obtenemos que la solución es
    \[
        y
        =
        x_0 \exp\left(-\frac{\tau}{2} \int_{0}^{t} ds\right)
        +
        \frac{\gamma}{2}
            \int_{0}^{t} \exp \left(-\frac{\tau}{2} \int_{s}^{t} ds\right) ds
        =
        x_0 e^{-\frac{\tau t}{2}}
        +
        \frac{\gamma}{\tau} \left(1 - e^{-\frac{\tau t}{2}}\right)
    .\]
    Como el sistema es autónomo y lineal, vale para todo tiempo \(t\) (ya que
    los coeficientes del sistema son continuos).

    Para \(x_0 < 0\) el mismo cambio de variables nos arroja la misma solución
    (reemplazando \(x_0\) con \(-x_0\)).

    \item Primero reescribamos la ecuación en términos de \(x\)
    \[
        x =
        \pm
        \left(
            \abs{x_0} e^{-\frac{\tau t}{2}}
            +
            \frac{\gamma}{\tau} \left(1 - e^{-\frac{\tau t}{2}}\right)
        \right)^2
    .\]

    Podemos notar múltiples soluciones para \(x_0 = 0\), esto no contradice
    Picard-Lindelöf pues la función \(f(t,x) = \gamma \sqrt{\abs{x}} - \tau x\)
    no es Lipschitz en \(x\) cerca de \(0\), en efecto \(\frac{\gamma \sqrt{\abs{x}}
    - \tau x - 0}{x - 0}\) no es acotada.

    \item Para \(x_0 > 0\), tomando el límite vemos que
    \[
        \lim_{t\to\infty} x(t) = \pm \left(\frac{\gamma}{\tau}\right)^2
    .\]
    Así que dependiendo de la solución que escogamos, esta se pegará a
    \((\gamma/\tau)^2\) o \(-(\gamma/\tau)^2\).
\end{plist}
\end{proof}

% PROBLEMA 5
\begin{problema}
Considere el problema de valor inicial
\[
    \dot{x} = x^3 - e^{t^2} x^2,\qquad x(0) = \xi
    \qquad\qquad (t,x) \in [0,\infty)\times \mathbb{R}
.\]
\begin{plist}
    \item Identifique y dibuje la 0-isoclina de la ecuación.

    \item Demuestre que para \(\xi \in [0,1]\), la solución \(x(t)\) está
    definida para todos \(t \ge 0\) y que \(\lim_{t \to \infty} x(t) = 0\).

    \item Pruebe que cuando \(\xi \ge \xi_0\) es suficientemente grande,
    \(x(t)\) explota en tiempo finito. (\textit{Sugerencia:} Construya una
    subsolución de la forma \(y(t) = e^{t^2} g(t)\) que explota en tiempo
    finito).
\end{plist}
\end{problema}
\begin{proof}
\begin{plist}
    \item LLamemos \(f(t,x)\) al lado derecho de la ecuación.
    Luego, la 0-isoclina corresponde a la curva de nivel \(f(t,x) = 0\).
    \[
        \left\{ f(t,x) = 0 \right\}
        =
        \left\{
            x = 0
        \right\}
        \cup
        \left\{
            x = e^{t^2}
        \right\}
    .\]
    \begin{figure}[H]
    \centering
    \begin{tikzpicture}
        \def\dmin{0};
        \def\dmax{1.1};

        \pgfmathsetmacro\rmin{\dmin-.5};
        \pgfmathsetmacro\rmax{e^(\dmax^2)};

        \clip (\rmin, {\rmax+.1}) rectangle ({2*\dmax +.1}, 2*\rmin);

        \draw[lightgray,dashed,step=1cm] (\dmin, \rmax) grid ({2*\dmax},
        {2*\rmin});

        \draw[->] (0, \rmin) -- (0, \rmax) node[left] {\(x\)};
        \draw[->] (\rmin, 0) -- ({2*\dmax}, 0) node[below] {\(t\)};

        \draw[thick, blue] (\rmin, 0) -- ({2*\dmax}, 0);

        \draw[domain=\dmin:\dmax, smooth, variable=\t, red]
            plot ({\t}, {e^(\t^2)});

        \node[fill=white] at (.3, 1.8) {+};
        \node[fill=white] at (.3, .5) {-};
        \node[fill=white] at (.3, -.5) {-};
    \end{tikzpicture}
    \caption{En rojo la curva \(x = e^{t^2}\) y en azul \(x = 0\)}
    \end{figure}

    \item Primero veamos los bordes. Para \(\xi = 0\), la derivada es nula en
    toda la vecindad y no tiene escapatoria así que se queda dentro de la
    solución \(x=0\). Para \(\xi = 1\), la derivada también es nula, sin
    embargo, en la vecindad la derivada es negativa, luego, la solución que pasa
    por aquí debe bajar. Así mismo, para \(\xi \in (0,1)\) la derivada es
    negativa así que la solución debe bajar. Dado que en \(x = 0\) tenemos una
    0-isoclina/asíntota, las soluciones se estancan en este valor, esto a su vez
    no asegura que ninguna solución explota en tiempo finito.

    \item Consideremos \(\xi \ge \xi_0 \gg 1\) (muy sobre la zona roja),
    aquí la derivada es positiva y la curva \(e^{t^2}\) es una subsolución.
    Necesitamos una subsolución que explote en tiempo finito.
    Consideremos
    \[
        y(t) = - \xi_0 \frac{e^{t^2}}{t-1}
        \qquad
        0 \le t < 1
    .\]
    Luego, \(y(0) = \xi_0\) y
    \[
        \dot{y} =
        -\xi_0 e^{t^2} \left(\frac{2t}{t-1} - \frac{1}{(t-1)^2}\right)
        <
        f(t,y) =
        -\xi_{0} e^{t^2} \left(
            \frac{\xi_0^2 e^{2t^2}}{(t-1)^3} - \frac{e^{2t^2}}{(t-1)^2}
        \right)
    .\]
    Así que, en efecto, es una subsolución y explota en \(t=1\), por lo tanto
    cualquier otra solución con PVI sobre \(\xi_0\) explota en tiempo finito.
\end{plist}
\end{proof}

% PROBLEMA 6
\begin{problema}[Bifurcaciones]
Describa cómo cambia el retrato de fase correspondiente a la ecuación dada
cuando varía el parámetro \(r\in \mathbb{R}\) (ie. dibuje el diagrama de
bifurcaciones). Clasifique las bifurcaciones que ocurren y calcule el(los)
valor(es) de bifuración de \(r\):
\begin{plist}
    \item \(\dot{x} = rx - \frac{x}{1+x^2}\).
    \item \(\dot{x} = x^3 - 2x^2 - 3x - rx\).
\end{plist}
\end{problema}
\begin{proof}
\begin{enumerate}[
label=(\alph*),topsep=0pt,itemsep=0pt,leftmargin=1.5em,rightmargin=1em]
\item \(\dot{x} = rx - \frac{x}{1+x^2}\) \\
   Grafiquemos la curva de nivel \(f(x,r) = 0\). Esto es, el conjunto
   \[
       \left\{ rx - \frac{x}{1+x^2} = 0 \right\}
       =
       \left\{ x = 0 \right\}
       \cup
       \left\{ r = \frac{1}{1+x^2} \right\}
   .\]

\sideToSide{.45}%
{%
    \begin{figure}[H]
    \caption{Gráfico de la curva de nivel \(f(x,y) = 0\).}
    \centering
    \begin{tikzpicture}
    \begin{axis}[
        grid=both,
        grid style={very thin, gray!20!white},
        axis lines = middle,
        xlabel = {\(x\)},
        ylabel = {\(r\)},
        xmin=-2, xmax=2,
        ymin=-1, ymax=2]

    \addplot[red,samples=500,thick] {1/(1+x^2)};
    \addplot[red,thick] coordinates {(0,-2)(0,2)};

    \node[label=Positivo] at (axis cs: 1,1) {};
    \node[label=Positivo] at (axis cs: -1,-.7) {};
    \node[label=Negativo] at (axis cs: -1,1) {};
    \node[label=Negativo] at (axis cs: 1,-.7) {};
    \end{axis}
    \end{tikzpicture}
    \end{figure}
}%
{%
    Luego la clasificación de puntos es
    \begin{itemize}[itemsep=0pt, leftmargin=2em, rightmargin=2em]
        \item Si \(r \le 0\) hay un solo punto de equilibrio y es estable.
        \item Para \(0 < r < 1\) hay tres puntos de equilibro. De izquieda a
        derecha (según el gráfico) tenemos la secuencia
        {\it inestable-estable-inestable}.
        \item Para \(r\ge 1\) tenemos un solo punto de equilibrio y es
        inestable.
    \end{itemize}
    La bifurcación es de tipo Pitchfork con \(r_c = 1\).
}

    \item \(\dot{x} = x^3 - 2x^2 - 3x - rx\).
    Graficamos la curva de nivel de \(f(x,r)\).
    \[
        \left\{ f(x,r) = 0 \right\}
        =
        \left\{ x = 0 \right\}
        \cup
        \left\{ r = (x-3)(x+1) \right\}
    .\]

\sideToSide{.5}%
{
    \begin{figure}[H]
    \centering
    \begin{tikzpicture}
    \begin{axis}[
        grid=both,
        grid style={very thin, gray!20!white},
        axis lines = middle,
        xlabel = {\(x\)},
        ylabel = {\(r\)},
        xmin=-5, xmax=5,
        ymin=-5, ymax=5]

    \addplot[red,samples=500,thick] {(x-3)*(x+1)};
    \addplot[red,thick] coordinates {(0,-5)(0,5)};

    \node[label=Negativo] at (axis cs: 1.5,1.5) {};
    \node[label=Negativo] at (axis cs: -3,-2.5) {};
    \node[label=Positivo] at (axis cs: 4,-4) {};
    \node[label={[rotate=-60]:Positivo}] at (axis cs: -1.5,3) {};
    \end{axis}
    \end{tikzpicture}
    \caption{Gráfico de la curva de nivel \(f(x,y) = 0\).}
    \end{figure}
}%
{%
    Luego, la clasificación de puntos es
    \begin{itemize}[itemsep=0pt, leftmargin=2em, rightmargin=2em]
        \item Si \(r < -4\) hay un solo punto de equilibrio y es {\it inestable}.
        \item Si \(r = -4\) hay dos puntos de equilibrio y ambos son {\it
        inestables}.
        \item Si \(-4 < r < -3\) hay tres puntos de equilibrio y siguen la
        secuencia {\it inestable-estable-inestable}.
        \item Si \(r = -3\) hay dos puntos de equilibrio y son
        {\it semiestable-inestable}.
        \item Si \(r > -3\) hay tres puntos de equilibrio y siguen la secuencia
        {\it inestable-estable-inestable}.
    \end{itemize}
    La bifurcación es de tipo transcrítico con \(r_c = -3\).
}
\end{enumerate}
\end{proof}

% PROBLEMA 7
\begin{problema}[Efecto Alle]
Para ciertas especies de animales, la razón de crecimiento efectiva de su
población, \(\dot{x}/x\), es máximal cuando la población llega a cierto nivel
positivo \(x = b > 0\). Una manera de modelar este fenómeno es por la ecuación
\[
    \frac{\dot{x}}{x} = r\left(a - (x/b - 1)^2\right),
    \qquad a,b,r > 0
.\]
Determine las condiciones necesarias y suficientes para los parámetros positivos
\(r,a,b\) para que exista un crítico nivel de población inicial \(x_c > 0\) tal que
si \(0 < x(0) < x_c\), la población se extingue a largo plazo.
\end{problema}
\begin{proof}
Llamemos \(f(x,r,a,b)\) a la función del lado derecho. Veamos cuando es cero.
\[
    \left\{ r \left(a - \left(\frac{x}{b} -1\right)^2\right) = 0 \right\}
    \overset{r>0}{=}
    \left\{ a = \left(\frac{x}{b} - 1\right)^2 \right\}
.\]
Los puntos fijos están dados por la intersección de la recta \(a\) y la parábola
\(((x/b)-1)^2\).

\sideToSide{.5}%
{%
Vemos que para \(a>0\) y \(b>0\) siempre habrán dos puntos fijos, llamemoslos
\(x^{-}\) y \(x^{+}\) respectivamente (\(x^{-} \le x^{+}\)). Luego,
\begin{clist}
\item Para \(x < x^{-}\), la parabola está sobre la recta, así que la derivada
es negativa.
\item Para \(x^{-} < x < x^{+}\), la recta está sobre la parábola, así que la
derivada es positiva.
\item Para \(x > x^{+}\), de nuevo la recta está bajo la parábola, así que la
derivada es negativa.
\end{clist}
Concluimos entonces que \(x^{-}\) es un punto inestable y \(x^{+}\) es un punto
estable.

Despejando obtenemos \(x^{-}\) explícitamente (\(x^{-} = b(1-\sqrt{a})\)) y por
el análisis anterior vemos que para \(x(0) < x^{-} = x_c\) la población
decrecerá hasta extinguirse. La condición necesaria es que el sistema tenga dos
soluciones positivas distintas, esto para si y solamente si \(\sqrt{a} < 1\).
}%
{%
\begin{figure}[H]
\centering
\begin{tikzpicture}
\def\xmin{0};
\def\xmax{4};
\def\ymin{-.5};
\def\ymax{2};

\def\a{.5};
\def\b{1};
\begin{axis}[
    grid=both,
    grid style={very thin, gray!20!white},
    axis lines = middle,
    xlabel = {\(x\)},
    ylabel = {\(y\)},
    xmin=\xmin, xmax=\xmax,
    ymin=\ymin, ymax=\ymax]

    \addplot[red,samples=500] {(x/\b - 1)^2};
    \addplot[red] coordinates {(\xmin,\a)(\xmax,\a)};

    \draw[dashed]
        (axis cs: {(1-sqrt(\a))*\b}, \a) -- (axis cs: {(1-sqrt(\a))*\b},0)
        node[below] {\(x^{-}\)};
    \draw[dashed]
        (axis cs: {(1+sqrt(\a))*\b}, \a) -- (axis cs: {(1+sqrt(\a))*\b},0)
        node[below] {\(x^{+}\)};
\end{axis}
\end{tikzpicture}
\parbox{.7\linewidth}{
\caption{Puntos fijos de la curva de nivel \(f(x,r,a,b) = 0\) para \(a=.5, b=1\).}
}
\end{figure}
}%
\end{proof}

% PROBLEMA 8
\begin{problema}
Sean \(T>0\), \(M\ge 1\), y sea \(a\colon \left[0,T\right] \to \mathbb{R}^{n
\times n}\) una función continua con valores en matrices \(n \times n\), que
satisface
\[
    \abs{a(t)v} \le M\abs{v}
    \qquad
    \forall v\in \mathbb{R}^{n}, t\in \left[0,T\right]
.\]
Demuestre que si \(x\in C^2 \left(\left[0,T\right],\mathbb{R}^n\right)\) es una
solución de problema de segundo orden
\[
    \begin{cases}
        \ddot{x} = a(t) x &, (t,x) \in [0,T]\times \mathbb{R}^n,\\
        x(0) = x_0, \dot{x}(0) = y_0 &, x_0, y_0 \in \mathbb{R}^n
    \end{cases}
\]
entonces \(x(t)\) satisface la estimación
\[
   \abs{x(t)}
   \le
   \sqrt{\abs{x_0}^2 + \abs{y_0}^2} e^{Mt}
   \qquad
   \forall t\in [0,T]
.\]
\end{problema}
\begin{proof}
Poniendo \(u = (u_1 = x, u_2 = \dot{x})\) nos queda el sistema equivalente
\begin{equation*}\label{p8:eq1}
    \dot{u} = (\dot{x}, a(t) x) = f(u)
    \quad,\quad
    u(0) = (x_0, y_0) = u_0
    \tag{\(\dag\)}
\end{equation*}
Notemos que \(\abs{u_0} = \sqrt{\abs{x_0}^2 + \abs{y_0}^2}\). Luego, por el TFC
tenemos que
\[
    u - u_0 = \int_{0}^{t} \dot{u}(s) ds
.\]
Tomando norma tenemos que
\[
    \abs{u} \le \abs{u_0} + \int_{0}^{t} \abs{\dot{u}(s)} ds
.\]
Notemos que
\[
    \abs{\dot{u}(s)} = \sqrt{\abs{\dot{x}(s)}^2 + \abs{a(t)x}^2}
.\]
Y si dios quiere de aquí sale usando Grönwell en algún momento.

\end{proof}
\end{document}
