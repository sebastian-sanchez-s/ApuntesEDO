\documentclass[../pheader.tex]{subfiles}

\begin{document}
{\sc Ayudantía 3 - EDO\hfill \small \rm Apuntes: Sebastián Sánchez}

\begin{center}
    \begin{tabular}{rl}
        Docente:& Nikola Kamburov\\
        Ayudante:& Jorge Acuña
    \end{tabular}
\end{center}

% P1
\begin{problema}
Sea \(t\in [a,b]\), \(p(t)\) una matriz \(n\times n\) con entradas continuas, y
\(q\colon [a,b] \to \mathbb{R}^n\) continua. Demuestre que la siguiente EDO
tiene una única solución para toda condición inicial \(y(t_0) = y_0 \in
\mathbb{R}^n\).
\[
    y'(t) + p(t) y = q(t)
.\]
\end{problema}
% R1
\begin{proof}
Supongamos que hay dos soluciones distintas, \(y(t)\) y \(z(t)\). Definamos
\(x(t) = y(t) - z(t)\), entonces \(x(t)\) satisface el PVI
\[
    \dot{x} = - p(t) x\quad,\quad x(t_0) = 0
.\]
Integrando tenemos que
\[
    x(t) = -\int_{t_0}^{t} p(s) x(s) ds
    \implies
    \abs{x(t)} \le \int_{t_0}^{t} \abs{p(s) x(s)} ds
.\]
Dado que \(t\in [a,b]\) y \(p(s)\) tiene entradas continuas, en particular es
acotada. Luego
\[
    \abs{x(t)} \le \int_{t_0}^{t} K\abs{x(s)}ds
\]
y por Grönwall tenemos que
\[
    \abs{x(t)} \le 0
.\]
Así que \(x \equiv 0\) y por lo tanto las soluciones son idénticas.
\end{proof}

% P2
\begin{problema}
Suponga que \(f\in \mathcal{C}^\infty \left(U, \mathbb{R}\right)\) satisface
\[
    \abs{f(t,x) - f(t,y)}
    \le
    L(t)\abs{x-y}
.\]
Muestre que la solución \(\phi(t,x_0)\) del problema
\[
    \dot{x} = f(t,x),\quad x(t_0) = x_0
\]
satisface
\[
    \abs{\phi(t,x_0) - \phi(t,y_0)}
    \le
    \abs{x_0 - y_0} \exp\left(\int_{t_0}^{t} \abs{L(s)} ds\right)
.\]
\end{problema}
% R2
\begin{proof}
Sea \(x = \phi(t,x_0)\) y \(y = \phi(t,y_0)\) dos soluciones del PVI respectivo.
Sabemos que
\[
    \dot{x} = f(t,x) \quad x(t_0) = x_0
    \implies
    x = x_0 + \int_{t_0}^{t} f(s,x(s)) ds
\]
ciertamente tenemos la misma expresión para \(y\). Se sigue que
\[
    x-y = x_0 - y_0 + \int_{t_0}^{t} f(s,x(s)) - f(s,y(s)) ds
\]
y tomando norma se obtiene
\begin{align*}
    \abs{x-y}
    &\le
    \abs{x_0 - y_0}
    +
    \abs{\int_{t_0}^{t} f(s,x(s)) - f(s,y(s)) ds}
    \\&\le
    \abs{x_0 - y_0}
    +
    \int_{t_0}^{t} \abs{L(s)} \abs{x(s) - y(s)} ds
\end{align*}
Finalmente, como \(\abs{x-y}\), \(\abs{L(s)} > 0\) son continuas se cumplen las
hipótesis para usar Grönwall, así que:
\[
    \abs{x-y} \le \abs{x_0 - y_0} \exp\left(\int_{t_0}^{t} \abs{L(s)} ds \right)
.\]
\end{proof}

% P3
\begin{problema}
Sea \(f\in \mathcal{C}(U, \mathbb{R})\), donde \(U\) es un subconjunto abierto de
\(\mathbb{R}^{n+1}\). Sea \((t_0, x_0) \in U\), y \(T,\delta > 0\) tal que
\(\left[t_0, t_0 + T\right] \times \overline{B_{\delta}(x_0)} \subset U\).
Definimos
\[
    L(t) = \sup \left\{
        \frac
            {\abs{f(t,x) - f(t,y)}}
            {\abs{x-y}}
        \colon
        x\ne y \in B_{\delta}(x_0)
    \right\}
.\]
Muestre que si \(L(t)\) es localmente absolutamente integrable en \(t\) (es
decir, \(\int_{I} \abs{L(t)}dt < \infty\) para todo intervalo compacto \(I\))
entonces existe una única solución local \(\tilde{x}(t) \in \mathcal{C}(I)\) del
problema de Cauchy (PVI)
\[
    \dot{x} = f(t,x), \quad x(t_0) = x_0
\]
donde \(I\) es algún intervalo alrededor de \(t_0\).
\end{problema}
% R3
\begin{proof}
Sean \(x, y\) dos soluciones del PVI. Se sigue que
\[
    \begin{cases}
        x = x_0 + \int_{t_0}^{t} f(s,x(s)) ds\\
        y = x_0 + \int_{t_0}^{t} f(s,y(s)) ds
    \end{cases}
    \implies
    x-y = \int_{t_0}^{t} f(s,x(s)) - f(s,y(s)) ds
.\]
Definamos el operador \(K(\phi)(t) = x_0 + \int_{t_0}^{t} f(s,x(s)) ds\). Luego,
la expresión anterior la podemos expresar como
\[
    \begin{cases}
        K(x)(t) = x_0 + \int_{t_0}^{t} f(s,x(s)) ds\\
        K(x)(t) = x_0 + \int_{t_0}^{t} f(s,y(s)) ds
    \end{cases}
    \implies
    K(x) - K(y) = \int_{t_0}^{t} f(s,x(s)) - f(s,y(s)) ds
\]
que tomando norma nos deja con y multiplicando por \(1\) en l integrando nos
deja:
\[
    \abs{K(x) - K(y)}
    \le
    \int_{t_0}^{t} L(s) \abs{x(s) - y(s)} ds
.\]

\end{proof}
% P4
\begin{problema}
Considere
\[
    f(y) =
    \begin{cases}
        4(y-1)^{3/4} &, y \ge 1\\
        0 &, y < 1
    \end{cases}
\]
\begin{plist}
\item Demuestre que el PVI
\[
    \frac{dy}{dt} = f(y), \quad y(0) = 2
\]
tiene única solución local y encuentre el intervalo máximo de su definición.
¿Es la solución maximal única?

\item Para cada \(\varepsilon > 0\) fijo, construya una familia infinita de
soluciones del problema
\[
    \frac{dy}{dt} = f(y),\quad y(0) = 1
\]
definidas en todo \(\mathbb{R}\) y distintas en el intervalo
\(\left(-\varepsilon, \varepsilon\right)\). ¿Por qué este resultado no
contradice el teorema de Picard-Lindelöf?
\end{plist}
\end{problema}
% R4
\end{document}
