\documentclass[10pt]{article}

\usepackage[margin=.7in]{geometry}
\usepackage{enumitem}
\usepackage{multicol}
\usepackage{float}
\usepackage{multicol}
\usepackage{tikz,pgfplots}
\usepackage{amsmath,amsthm,amssymb}
\usepackage{mathtools}
\usepackage[utf8]{inputenc}
\usepackage[spanish]{babel}
\usetikzlibrary{babel}

\usepackage{mdframed}

\usepackage{amstext} % for \text macro
\usepackage{array}   % for \newcolumntype macro
\newcolumntype{C}{>{$}c<{$}} % math-mode version of "c" column type
\newcolumntype{L}{>{$}l<{$}} % math-mode version of "l" column type

\newcounter{problemctr}[section]
\newenvironment{problema}{
    \refstepcounter{problemctr}
    \mdfsetup{%
        frametitle={\tikz\node[fill=white,rectangle,inner sep=1pt,outer
        sep=1pt]{Problema \theproblemctr};},
        frametitleaboveskip=-0.5\ht\strutbox,
        frametitlealignment=\center
    }%
    \begin{mdframed}[style=exampledefault]
    }{\end{mdframed}
}

\newlist{plist}{enumerate}{1}
\setlist[plist]{label=(\alph*),leftmargin=2em,rightmargin=2em,itemindent=2pt,itemsep=0pt,topsep=0pt}

\newlist{clist}{itemize}{1}
\setlist[clist]{label=\(\bullet\), listparindent=1em, leftmargin=2pt,itemindent=2pt}

\usepackage[T1]{fontenc}

%\usepackage{subfiles}

% Resize abs and norm
\DeclarePairedDelimiter{\abs}{\lvert}{\rvert}
\DeclarePairedDelimiter{\norm}{\|}{\|}
\makeatletter
\let\oldabs\abs
\def\abs{\@ifstar{\oldabs}{\oldabs*}}
\let\oldnorm\norm
\def\norm{\@ifstar{\oldnorm}{\oldnorm*}}
\makeatother

\newcommand\sideToSide[3]%
{
    \begin{minipage}{#1\textwidth}
       #2
    \end{minipage}
    \begin{minipage}{\dimexpr\textwidth-#1\textwidth}
       #3
    \end{minipage}
}

%\begin{document}
%    \subfile{Ayudantias/A5}
%    \subfile{Ayudantias/A10}
%\end{document}



\newgeometry{margin=1in}

\begin{document}

{\sc Ayudantía 10 - EDO \hfill \small \rm Apuntes: Sebastian Sánchez}

\begin{center}
    \begin{tabular}{rl}
        Docente:& Nikola Kamburov\\
        Ayudante:& Jorge Acuña
    \end{tabular}
\end{center}

% P1
\begin{problema}
    \begin{enumerate}
        \item
        Considere la ecuación
        \[
            P(x)y'' + Q(x)y' + R(x)y = -zy
        .\]
        Escríbala en la forma de Sturm-Liouville
        \[
            -(p(x)y')' + (q(x)-z\,r(x))y = 0
        \]
        donde \(z\in \mathbb{C}\).

        \item
        Encuentre la forma del operador de Sturm-Liouville para
        \[
            \begin{cases}
                -xy''-(1-x)y' = zy &, x\in (1,2)\\
                y(1) = y(2) = 0
            \end{cases}
        .\]
    \end{enumerate}
\end{problema}
\begin{enumerate}
\item
Multiplicando la ecuación por el factor integrante
\[
    \Phi(x) = \frac{1}{P(x)}\exp\left(\int_{t_0}^{x} \frac{Q(t)}{P(t)} dt\right)
\]
la ecuación se expresa como (abusando notación para fines estéticos)
\begin{align*}
    y''\,e^{\int Q/P} + \frac{Q}{P}\,e^{\int Q/P} y' + Ry\Phi &= -zy\Phi\\
    (y'\, e^{\int Q/P})' + Ry\phi &= -zy\Phi\\
    -(y'\, \underbrace{e^{\int Q/P}}_{p})' + (\underbrace{-R\phi}_{q} -
    z\underbrace{\Phi}_{r}) y &= 0\\
    -(py')' + (q-z\,r)y &= 0
\end{align*}
llegándose a lo querido.

\item
\end{enumerate}

% P2
\begin{problema}
\begin{enumerate}[leftmargin=1em,itemsep=.5em]
	\item
	Encuentre criterio sobre los parámetros \(\alpha, \beta \in \mathbb{R}\)
	para que el problema
	\[
		\begin{cases}
			y''(0) = 0 &, t\in (0,1)\\
			y'(0) = \alpha y(0) &, y'(1) = \beta y(1)
		\end{cases}
	\]
	tenga solo solución nula.

	\item
	Suponga que se cumple el criterio de la parte anterior. Encuentre función
	continua \(G\colon [0,1]^2 \to \mathbb{R}\) tal que para cada \(f\in
	\mathcal{C}[0,1]\) la solución al problema
	\[
		\begin{cases}
			-y'' = f &, t\in (0,1)\\
			y'(0) = \alpha y(0) &, y'(1) = \beta y(1)
		\end{cases}
	.\]
	se escribe como
	\[
		y(t) = \int_{0}^{1} G(t,s) f(s) ds, \quad t\in [0,1]
	.\]
\end{enumerate}
\end{problema}
% R2
\begin{enumerate}[leftmargin=1em,itemsep=.5em]
	\item % 2.1
	Como la segunda derivada de \(y\) es nula, esta debe ser una función lineal
	de la forma \(y(t) = at + b\) que asumiremos no nula (i.e. \(a\ne 0\)).

	Aplicando las condiciones de frontera obtenemos que
	\[
		\begin{cases}
			a = y'(0) = \alpha y(0) = \alpha b\\
			a = y'(1) = \beta y(1) = \beta (a+b)
		\end{cases}
	\]
	Notar que \(\alpha \ne -1\) y \(b\ne 0\), pues en esos caso \(a\) es
	necesariamente \(0\), lo cual no puede pasar ya que la suposición es que
	\(y\) es una función lineal no nula. Luego:
	\[
		\beta = \frac{\alpha}{\alpha+1}
	\]
	Así, para que \(y\) sea la función nula es necesario que \(\alpha = -1\) o
	bien \(\beta \ne \alpha/(\alpha+1)\).

	\item % 2.2
	El operador de Sturm-Liouville asociado al sistema es
	\[
		L = -\frac{d^2}{dx^2}; \hspace{.5cm}
		B_0 \colon y'(0) = \alpha y(0); \hspace{.5cm}
		B_1 \colon y'(1) = \beta y(1)
	.\]
	Por la parte anterior, si \(\beta \ne \alpha/(\alpha+1)\) el operador \(L-0 = L\)
	es invertible y la solución \(y\) está dada por la fórmula:
	\[
		y(t)
		=
		W^{-1}\, \left(
		\int_{0}^{t} u^{1}(t) u^{0}(s) f(s) ds
		+
		\int_{t}^{1} u^{0}(t) u^{1}(s) f(s) ds
		\right)
	.\]
	Donde \(W^{-1}\) es el Wronskiano modificado y \(u^{i}\) es una solución del
	del problema \(Lu^{i} = 0\) con condición iniciales \(B_i\).
	Resolviendo los problemas correspondientes obtenemos que \(u^{0}(t) = \alpha t + 1\) y \(u^{1}(t)
	= \beta (t-1) + 1\). De esta forma:
	\[
		W
		=
		u^{0}(t) \frac{d}{dt} u^{1}(t) - u^{1}(t) \frac{d}{dt} u^{0}(t)
		\overset{t=0}{=}
		\beta - (1-\beta)\alpha
		=
		\beta (1+\alpha) - \alpha
		\ne 0
	.\]
	Y la función \(G\) queda definida como
	\[
		G(t,s) = -W^{-1}\,
		\begin{cases}
			u^{1}(t) u^{0}(s) &, s \le t\\
			u^{0}(t) u^{1}(s) &, s \ge t\\
		\end{cases}
	.\]
\end{enumerate}

% P3
\begin{problema}
	Sea \(\theta \in \mathbb{S}^{1} \coloneqq \left\{ z\in \mathbb{C}\colon \abs{z} = 1
	\right\}\), \(q = \bar{q} \in \mathcal{C}\left[0,2\pi\right]\). Definamos el operador,
	\[
		L_{q,\theta} \coloneqq -\frac{d^2}{dt^2} + q
	\]
	con dominio \(\mathcal{D}(L_{q,\theta}) = \left\{ u\in
	\mathcal{C}^2\left[0,2\pi\right]\colon u(2\pi) = \theta u(0),\, u'(2\pi) = \theta
	u'(0)\right\}\).
	\begin{enumerate}
		\item Demuestre que \(L_{q,\theta}\) es simétrico en
		\(L^{2}\left(0,2\pi\right)\).
		\item Suponga que \(q(t) = 1\) para todo \(t\in \left[0,2\pi\right]\). Halle
		explícitamente los valores propios de \(L_{1,\theta}\) y las funciones
		propias correspondientes para \(\theta \in \mathbb{S}^{1}\). Encuentre
		el conjunto:
		\[
			\left\{ \theta \in \mathbb{S}^1 \colon \text{Todos los valores
			propios de \(L_{1,\theta}\) son simples}\right\}
		.\]
	\end{enumerate}
\end{problema}
\begin{enumerate}
	\item % P3.1
	Sean \(u, v \in \mathcal{D}(L)\), luego:
	\begin{align*}
		\langle v, L_{q,\theta} u\rangle - \langle L_{q,\theta} v, u\rangle
		&=
		\int_{0}^{2\pi} \overline{v(t)} (-u''(t) + u(t) q(t)) dt
		-
		\int_{0}^{2\pi} \overline{(-v''(t) + v(t) q(t))} u(t) dt
		\\&=
		\int_{0}^{2\pi} -\overline{v(t)} u''(t) dt
		-
		\int_{0}^{2\pi} - \overline{v''(t)} u(t) dt
	\end{align*}
\end{enumerate}
\end{document}
