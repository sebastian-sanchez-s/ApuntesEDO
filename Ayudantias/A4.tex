\documentclass[../pheader.tex]{subfiles}

\begin{document}
{\sc Ayudantía 4 - EDO \hfill \small \rm Apuntes: Sebastian Sánchez}

\begin{center}
    \begin{tabular}{rl}
        Docente:& Nikola Kamburov\\
        Ayudante:& Jorge Acuña
    \end{tabular}
\end{center}

% P1
\begin{problema}
Sea la EDO
\[
    \dot{y} = ty^2;\quad y(0) = x
.\]
donde su solución es \(\phi(t,x) = y(t)\). Encuentre las ecuaciones
variacionales para \(\partial_{x}\phi(t,x), \partial_{x}^2 \phi(t,x)\) y una
expresión explícita para \(\partial_{x}^2 \phi(t,x)\).
\end{problema}
Sabemos \(\phi(t,x)\) cumple la EDO, así que derivando con respecto a la
posición tenemos (notar que \(\phi \in \mathcal{C}^2\)):
\[\label{p1:ev1}
    \partial_t \partial_x \phi(t,x)
    =
    \partial_x \partial_t \phi(t,x)
    =
    \partial_x \left(t(\phi(t,x)^2\right)
    =
    2t\phi(t,x) \partial_x \phi(t,x)
    \tag{EV \(\partial_x \phi\)}
.\]
Derivando nuevamente con respecto a la posición obtenemos:
\[\label{p1:ev2}
    \partial_t \partial_x^2 \phi(t,x)
    =
    2t\left((\partial_x \phi(t,x))^2 + \phi(t,x) \cdot \partial_x^2 \phi(t,x)\right)
    \tag{EV \(\partial_x^2 \phi\)}
.\]

En \eqref{p1:ev1}, llamemos \(z(t) = \partial_x \phi(t,x)\). Esto nos deja con
el PVI:
\[
    \dot{z} = 2t\phi(t,x) z,\quad
    z(0) = 1
\]
el cual es separable, así que la solución viene dada por:
\[
    z(t) = \exp\left(\int_{0}^{t} 2s\phi(s,x) ds \right)
\]
como estamos trabajando en posiciones lejos de \(0\), podemos multiplicar el
integrando por \(\phi/\phi\) y usando la EDO original nos queda:
\[
    z(t)
    =
    \exp\left(\int_{0}^{t} \frac{2s\phi(s,x)^2}{\phi(s,x)} ds \right)
    =
    \exp\left(\int_{0}^{t} \frac{2 \delta_s \phi(s,x)}{\phi(s,x)} ds \right)
    =
    \abs{\frac{\phi(t,x)}{\phi(0,x)}}^2
    =
    \frac{\phi(t,x)^2}{x^2}
\]
deshaciendo el cambio de variables vemos que
\[
    \partial_x \phi(t,x)
    =
    \frac{\phi(t,x)^2}{x^2}
.\]

% P2
\begin{problema}
Muestre que en una dimensión, y si \(f\in \mathcal{C}^1\), se tiene
\[
    \frac{\partial\phi}{\partial x}(t,x)
    =
    \exp \left(\int_{t_0}^{t} \frac{\partial f}{\partial x}(s, \phi(s,x)) ds \right)
.\]
\end{problema}

% P3
\begin{problema}
Considere una ecuación de primer orden en \(\mathbb{R}^1\) con \(f(t,x)\colon
\mathbb{R}\times \mathbb{R}\to \mathbb{R}\) continua. Suponga que \(xf(t,x) <
0\) para \(\abs{x} > R\). Muestre que todas las soluciones existen para todo
\(t > 0\).
\end{problema}

% P4
\begin{problema}
Sea \(f\in \mathcal{C}\left(\mathbb{R}\times \mathbb{R}^{n},
\mathbb{R}^n\right)\) y
\[
    \abs{f(t,x)} \le g(\abs{x})
\]
para alguna función \(g\in \mathcal{C}([0,\infty)\) que satisface
\[
    \int_{0}^{\infty} \frac{dr}{g(r)} = \infty
.\]
Entonces todas las soluciones del PVI
\[
    \dot{x} = f(t,x); \quad x(t_0) = x_0
\]
están definidas para todo \(t\ge 0\).
\end{problema}
\end{document}
