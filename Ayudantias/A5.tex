\documentclass[10pt]{article}

\usepackage[margin=.7in]{geometry}
\usepackage{enumitem}
\usepackage{multicol}
\usepackage{float}
\usepackage{multicol}
\usepackage{tikz,pgfplots}
\usepackage{amsmath,amsthm,amssymb}
\usepackage{mathtools}
\usepackage[utf8]{inputenc}
\usepackage[spanish]{babel}
\usetikzlibrary{babel}

\usepackage{mdframed}

\usepackage{amstext} % for \text macro
\usepackage{array}   % for \newcolumntype macro
\newcolumntype{C}{>{$}c<{$}} % math-mode version of "c" column type
\newcolumntype{L}{>{$}l<{$}} % math-mode version of "l" column type

\newcounter{problemctr}[section]
\newenvironment{problema}{
    \refstepcounter{problemctr}
    \mdfsetup{%
        frametitle={\tikz\node[fill=white,rectangle,inner sep=1pt,outer
        sep=1pt]{Problema \theproblemctr};},
        frametitleaboveskip=-0.5\ht\strutbox,
        frametitlealignment=\center
    }%
    \begin{mdframed}[style=exampledefault]
    }{\end{mdframed}
}

\newlist{plist}{enumerate}{1}
\setlist[plist]{label=(\alph*),leftmargin=2em,rightmargin=2em,itemindent=2pt,itemsep=0pt,topsep=0pt}

\newlist{clist}{itemize}{1}
\setlist[clist]{label=\(\bullet\), listparindent=1em, leftmargin=2pt,itemindent=2pt}

\usepackage[T1]{fontenc}

%\usepackage{subfiles}

% Resize abs and norm
\DeclarePairedDelimiter{\abs}{\lvert}{\rvert}
\DeclarePairedDelimiter{\norm}{\|}{\|}
\makeatletter
\let\oldabs\abs
\def\abs{\@ifstar{\oldabs}{\oldabs*}}
\let\oldnorm\norm
\def\norm{\@ifstar{\oldnorm}{\oldnorm*}}
\makeatother

\newcommand\sideToSide[3]%
{
    \begin{minipage}{#1\textwidth}
       #2
    \end{minipage}
    \begin{minipage}{\dimexpr\textwidth-#1\textwidth}
       #3
    \end{minipage}
}

%\begin{document}
%    \subfile{Ayudantias/A5}
%    \subfile{Ayudantias/A10}
%\end{document}



%\newgeometry{margin=1.3in}

\begin{document}

{\sc Ayudantía 5 - EDO \hfill \small \rm Apuntes: Sebastian Sánchez}

\begin{center}
    \begin{tabular}{rl}
        Docente:& Nikola Kamburov\\
        Ayudante:& Jorge Acuña
    \end{tabular}
\end{center}

% P1
\begin{problema}
Halle la solución del siguiente sistema de ecuaciones diferenciales
\[
    y'
    =
    \begin{pmatrix}
    1 & 1 & 1 \\
    1 & 1 & 1 \\
    1 & 1 & 1
    \end{pmatrix}
    y
.\]
\end{problema}
% R1
Sabemos que la solución al sistema viene dada por el propagador \(\exp(tA)\).
Recordando que:
\[
    \exp(tA) = \sum_{n=0}^\infty \frac{t^n A^n}{n!}
\]
y notando que \(A^n = 3^{n-1} A\) nos da:
\begin{align*}
    \exp(tA)
    &=
    I + \frac{1}{3} \sum_{n=1}^{\infty} \frac{(3t)^n }{n!} A
    \\&=
    I + \frac{1}{3} \sum_{n=0}^{\infty} \frac{(3t)^n }{n!} A - \frac{1}{3} A
    \\&= I + \frac{1}{3} e^{3t} A  - \frac{1}{3} A
\end{align*}

% P2
\begin{problema}
Encuentre explícitamente la solución del sistema
\[
    y'(t) =
    \begin{pmatrix}
    3 & 1\\
    -1 & 5
    \end{pmatrix} y(t)
    +
    t
    \begin{pmatrix} e^{-4t} \\ e^{-4t} \end{pmatrix}
.\]
\end{problema}
% R2
Resolveremos el sistema homogéneo y luego usaremos la fórmula de Duhamel.

Veamos el caso homogéneo. La solución aquí está dada por el propagador
\(e^{tA}\) donde
\[
    A =
    \begin{pmatrix}
    3 & 1\\
    -1 & 5
    \end{pmatrix}
.\]
Vamos a calcular la forma canónica de Jordan. Para los valores propios vemos el
polinomio característico
\[
    \chi(z)
    =
    \det(zI - A)
    =
    (z-3)(z-5) + 1
    =
    (z-4)^2
.\]
Así que el único valor propio es \(4\) con multiplicidad 2. Para los vectores
propios
\begin{align*}
    Av_1 = 4v_1 &\implies (1,1)\\
    (A-4I)v_2 = v_1 &\implies (0,1)
\end{align*}
De esta forma,
\[
    A
    =
    \underbrace{\begin{pmatrix} 1 & 0\\ 1 & 1 \end{pmatrix}}_{V}
    \underbrace{\begin{pmatrix} 4 & 1\\ 0 & 4 \end{pmatrix}}_{A_j}
    \underbrace{\begin{pmatrix} 1 & 0\\-1 & 1 \end{pmatrix}}_{V^{-1}}
.\]
Así que
\[
    y_h(t) =
    V
    \begin{pmatrix} e^{4t} & t e^{4t}\\ 0 & e^{4t} \end{pmatrix}
    V^{-1}
    =
    e^{4t} \begin{pmatrix} 1-t & t \\ 2-t & t+1 \end{pmatrix}
.\]

Aplicando la fórmula de Duhamel obtenemos que
\begin{align*}
    y(t)
    &=
    \int_{t_0}^{t} e^{(t-s)A} f(s) ds + e^{(t-t_0)A} y_0
    \\&=
    \int_{t_0}^{t}
    \begin{pmatrix} 1-t+s & t-s\\ 2-t+s & t-s+1\end{pmatrix}
    \begin{pmatrix} e^{-4s} \\ e^{-4s} \end{pmatrix}
    ds
    +
    e^{(t-t_0)A}y_0
    \\&=
    \int_{t_0}^{t} e^{-4s} \begin{pmatrix} 1\\ 3 \end{pmatrix}ds
    +
    \begin{pmatrix}
    1-t+t_0 & t-t_0\\
    2-t+t_0 & t-t_0+1
    \end{pmatrix}
    y_0
\end{align*}

% P3
\begin{problema}
Considere el sistema:
\[
    \dot{x}_1 = x_2
    \hspace{1cm}
    \dot{x}_2 =
    \begin{cases}
        \frac{6 x_1}{t^2} &, t\ne0\\
        0 &, t=0
    \end{cases}
.\]
Demuestre que
\[
    f_1(t) = (t^3, 3t^2)
    \hspace{1cm}
    f_2(t) = (\abs{t}^3, 3t^2\sigma(t))
\]
son soluciones del sistema (\(\sigma\) es la función signo).

Muestre que \(\left\{ f_1, f_2 \right\}\) son linealmente independientes en
\(\mathbb{R}\), pero \(W(t) = 0\) para todo \(t\in \mathbb{R}\).
¿Por qué esto no contradice el criterio de independencia lineal de las
soluciones del sistema?
\end{problema}
% R3

% P4
\begin{problema}
Sean \(A,B \in \mathbb{R}^{k\times k}\) con \(k\in \mathbb{N}\). Demuestre que,
para \(t\in \mathbb{R}\):
\[
    e^{t(A+b)} = e^{tA}e^{tB}
    \iff
    AB = BA
.\]
\end{problema}
\end{document}
